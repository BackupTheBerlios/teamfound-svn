\documentclass[german]{article}

\usepackage{listings}
\lstset{numbers=left, numberstyle=\tiny, numbersep=5pt}
\lstset{language=Java}
\usepackage{german}
\usepackage[latin1]{inputenc}
\usepackage[T1]{fontenc}

\begin{document}

\title{Teamfound}

\author{Jonas Heese, ...}

\date{M�rz 2006}

\maketitle

\section{Einleitung}

\begin{itemize}
\item Was ist Teamfound, wo kommt es her?
\item Was ist der Sinn?
\item Was gibt es bisher an vergleichbarer Technologie?
\item Was sind die Probleme vergleichbarer Technologie und was macht TF besser?
\end{itemize}

\section{Teamfound-Server}

Die Entscheidung f�r eine Plattform wurde implizit durch die Entscheidung f�r eine Bibliothek zum indexieren und durchsuchen von Dokumenten getroffen. Nachdem einige Tage nach derartigen Bibliotheken Ausschau gehalten wurde, ist sehr schnell deutlich geworden, das Lucene \footnote{www.lucene.de} vom Apache-Projekt\footnote{http://www.apache.org} die am weitest entwickelte Bibliothek ist. Ein weiterer Vorteil ist die aktive Community rund um Lucene, so wurde w�hrend unserer Entwicklung noch ein Sprung von Version 1.5 auf 1.9 mitgemacht. Andere Bibliotheken, zum Beispiel in Perl oder PHP boten keine ausreichende Dokumentation oder Erweiterbarkeit.
Die urspr�ngliche Lucene-Version ist in Java implementiert, ein weiteres Projekt bem�ht sich um dessen Umsetzung in C und mittlerweile (Am Ende des Projekts hat Zend Technologies\footnote{http://www.zend.com} einen PHP-Wrapper\footnote{http://framework.zend.com/manual/en/zend.search.html} f�r Lucene ver�ffentlicht.

Somit war die Wahl der Plattform auf Java gefallen, eine erste mehr als Proof-of-Concept anzusehende Version von Teamfound wurde mittels zwei Perl-CGI-Scripten und einem Systemaufruf der Lucene-Bibliothek umgesetzt. Dieses Vorgehen startet bei jeder Anfrage an den Webserver eine neue Java virtuel machine, was nat�rlich �berhaupt nicht effektiv ist. 

Daher fiel eine weitere Entscheidung, f�r eine Java-Servlet-Umsetzung, um die ben�tigte Umgebung f�r Anfragen an den Lucene-Index nicht f�r jedes Request neu zu schaffen. Somit besteht der Server aus einer 100\%igen Java-Umgebung und es entf�llt damit die Notwendigkeit f�r aufw�ndige und zeit-kostende Wrapper um mehrere Programmiersprachen zu verbinden.

\subsection{Kernbibliotheken}

\subsubsection{Apache Lucene}

Lucene ist - laut eigener Aussage - der harte Teil einer Suchmaschine. Diese Bibliothek erzeugt und durchsucht einen Volltextindex, k�mmert sich aber weder darum wo die Daten herkommen, wie die Ergebnisse pr�sentiert werden oder wie die Suchanfragen vom Benutzer an den Index kommen. Der Index ist hingegen bietet viele M�glichkeiten die eigenen Daten zu analysieren und in sogenannten Dokumenten in den Index zu speichern. Eigene $Analyzer$ sind mit wenig Aufwand umzusetzen (sofern die mitgelieferten nicht ausreichen), auch aufw�ndigere Parser sind m�glich, wie es zB. f�r HTML-Seiten notwendig ist.
Dar�berhinaus kann jedes Dokument im Index beliebige $Felder$ enthalten, ein Feld ist als Inhaltselement eines Dokuments zu verstehen (�berschrift, Inhalt, Beschreibung). Dabei k�nnen v�llig verschiedene Arten von Dokumenten in einem Index gespeichert werden sowie Suchanfragen direkt an ein oder mehrere Felder gestellt werden. Diese starke Erweiterbarkeit ist zum Beispiel bei der Implementierung von Dokumentenkategorien in TeamFound sehr wichtig gewesen (Siehe TODO: link einf�gen)

TODO: Query Syntax, beteiligte Arten von Klassen genauer ausf�hren (Analyzer, Document, Field, IndexReader, IndexWriter, Query, QueryParser, ....)

\subsubsection{HSQLDB}

\subsubsection{Weitere Bibliotheken}

jdom

\subsection{Architektur}

\subsubsection{Servlet}

Das Servlet ist der Einstiegspunkt f�r jegliche Anfragen an TeamFound. Dem Servlet kommt dabei die Rolle eines W�chters zu, welcher �berpr�ft ob notwendige Parameter f�r die einzelnen Anfragen vorhanden sind. Ist eine Anfrage g�ltig, wird diese an den sogenannten Controller weitergereicht, welcher die eigentlichen TeamFound-Komponenten steuert. Dieser Controller gibt immer ein Response-Objekt zur�ck, welches vom Servlet dann zB. als XML oder HTML serialisiert und als Antwort zum Client geschickt wird.

\subsubsection{Controller}

\subsubsection{Indexer}

\subsubsection{Response}

Antworten werden in TeamFound immer als XML generiert, diese XML-Repr�sentation kann bevor sie zum Client geschickt wird, noch mittels XSLT in eine geeignete HTML-Fassung gewandelt werden.
Eine Basisklasse Response �bernimmt dabei das Handling des eigentlichen JDOM-Document-Objektes, w�hrend die eigentlichen Antwort-Klassen, wie SearchResponse oder AddPageResponse ihre eigentlichen Inhalte an die Standardelemente einer Antwort anh�ngen (Siehe dazu mehr im Kapitel Protokoll TODO: LINK).

\subsection{Ablauf eines Requests}

Komplette Beschreibung, evtl. auch mit Diagramm

\section{Protokoll}

Jeweils Anfrage und Antworten

\begin{itemize}
\item Zielsetzung und Sinn des Protokolls
\item Umsetzung (Designentscheidungen, Technik,...)
\item Dokumentation (Wie siehts wirklich aus)
\end{itemize}

\section{Clients}

\subsection{Web-Client}

\subsection{Firefox-Toolbar}

\subsection{Internet Explorer-Toolbar}


\section{Projektablauf, organisation, Meilensteine}


\section{Fazit}

