\documentclass[german]{article}

\usepackage{listings}
\lstset{numbers=left, numberstyle=\tiny, numbersep=5pt}
\lstset{language=Java}
\usepackage{german}
\usepackage[latin1]{inputenc}
\usepackage[T1]{fontenc}

\begin{document}

\title{Teamfound}

\author{Jonas Heese, ...}

\date{M�rz 2006}

\maketitle

\section{Einleitung}

\begin{itemize}
\item Was ist Teamfound, wo kommt es her?
\item Was ist der Sinn?
\item Was gibt es bisher an vergleichbarer Technologie?
\item Was sind die Probleme vergleichbarer Technologie und was macht TF besser?
\end{itemize}

\section{Serverarchitektur}

\subsection{Technik}
\begin{itemize}
\item Welche Plattform und warum?
\\
Java, Servlets
\item Welche Kernbibliotheken und warum?
\\
Lucene, HSQLDB, (JDom)
\end{itemize}

\subsection{Architektur}
\begin{itemize}
\item Servlet
\item Controller
\\
-> DB
\\
-> Kategorien
\item Indexer
\item Response (auch: xml->xsl->html)
\end{itemize}

\subsection{Ablauf eines Requests}

Komplette Beschreibung, evtl. auch mit Diagramm

\section{Protokoll}

Jeweils Anfrage und Antworten

\begin{itemize}
\item Zielsetzung und Sinn des Protokolls
\item Umsetzung (Designentscheidungen, Technik,...)
\item Dokumentation (Wie siehts wirklich aus)
\end{itemize}

\section{Clients}

\subsection{Web-Client}

\subsection{Firefox-Toolbar}

\subsection{Internet Explorer-Toolbar}


\section{Projektablauf, organisation, Meilensteine}


\section{Fazit}

