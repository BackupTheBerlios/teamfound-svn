\chapter{Interface}
\section{Interface Milestone 2}
\label{interface2}
\subsection{Allgemein}
\subsubsection{Tests}
Zum Testen der Serverseitigen Interface-Implementation existiert folgende Webseite, die entsprechende Requests erstellen kann: 

\texttt{http://teamfound.berlios.de/test\_the\_server\_interface\_m2.html}
\\
\\
Zum Testen der Clientseitigen Interface-Implementation existiert folgendes Perl-Skript, welches den Parameter command auswertet und eine enstsprechende statische XML-Antwort zurueckgibt: 
\\
\texttt{http://teamfound.dyndns.org/interface-m2/testmilestone2.pl}
\\
Demo: 
\\
\texttt{http://teamfound.dyndns.org/interface-m2/testmilestone2.pl?command=getcategories}
\\
Das Skript liegt auch im SVN unter \texttt{teamfound/interface/milestone2/testmilestone2.pl}

\subsubsection{Alle Anfragen an den Server}

Alle Anfrage-Parameter werden in Form von HTTP GET oder POST Variablen �bertragen. Prinzipiell gibt es eine Unterscheidung, ob f"ur jedes Kommando eine eigene URL verwendet wird, oder ob das Kommando in Form einer weiteren HTTP-GET Variablen mit "ubertragen wird. Unsere Implementation des  Servers verwendet den zus"atzlichen HTTP-GET Parameter \texttt{command}. In diesem Interface haben wir aber beide M"oglichkeiten spezifiziert.

\begin{description}
\item[want=xml or html] Die zu erwartende Antwort soll in xml oder html-format sein (default soll html werden, da dann einfache Link-Clients m�glich sind) 

\item[version=2] Milestone-Version des Interfaces 

\item[command=search] Das Kommando das ausgef�hrt werden soll. Dieses Argument kann wegfallen wenn f�r jedes Kommando eine eigene, gleichnamige Anfrage-URL existiert. 
\end{description}

\subsubsection{HTML-Antwort}
Die Clients geben an ob Sie eine HTML oder eine XML Antwort erwarten.

Soll eine vollst�ndige HTML-Seite zur�ckgeben, die direkt im Browser angezeigt werden kann. Die Return-Values wie bei einer XML-Antwort m�ssen in diesem Fall nicht zur�ckgegeben werden, sondern der Text sollte gleich eine entsprechende Meldung beinhalten.

\subsubsection{XML-Antwort}
\begin{description}
\item[$<$response$>$] Umschliesst alle anderen xml-tags und gibt die XSD-Datei zum verifizieren der XML-Daten an 
\item[$<$interface-version$>$] Gibt immer die Interface-Version an, in der die Antwort des Servers formuliert wurde 
\item[$<$server$>$] Gibt Name und Versionsnummer des Servers an. Der Server der unter der URL https://developer.berlios.de/projects/teamfound entwickelt wird gibt den Namen "TeamFound" zur�ck. Clones d�rfen Ihren eigenen Namen nat�rlich frei w�hlen. 
\item[$<$addpage$>$ $<$search$>$ $<$addcategory$>$ $<$getcategories$>$] Diese Tags beinhalten die Antworten auf die gleichnamigen Anfragen an den Server. Pro Anfrage darf nur eines dieser Tags vorkommen. 
\item[$<$return-value$>$ $<$return-description$>$] Der Return-Value bzw. Fehler-Code falls etwas schiefging. Die Description wird nicht genauer spezifiziert, sollte aber semantisch mit dem aufgetretenen Fehler �bereinstimmen. 
\item[$<$project-counter$>$] Bei jeder �nderung des Kategorien-Baumes( soll der Server einen internen Z�hler um eins inkrementieren. Der aktuelle Wert soll bei jeder XML-Antwort mit �bertragen werden. (Dann weiss die Toolbar wann sie selber die Kategorien neu vom Server abfragen muss.)Jedes Projekt fuehrt einen Counter, somit muss der Client nur die fuer ihn intressanten B"aume beachten. 
\end{description}

\begin{verbatim}
<response xmlns:xsi="http://www.w3.org/2001/XMLSchema-instance"
xsi:noNamespaceSchemaLocation="teamfound-interface-milestone2.xsd">

 <interface-version>2</interface-version>

 <return-value>0</return-value>
 <return-description>OK</return-description>

 <project-counter>
   <project>
     <projectID>0</projectID>
     <count>3</count>
   </project>
   <project>
     <projectID>3</projectID>
     <count>5</count>
   </project>
 </project-counter>

 <category-counter>54</category-counter>

 <server>
  <name>TeamFound</name>
  <version>0.2</version>
 </server>

 <addpage>
 </addpage>

 <search>
 </search>

 <addcategory>
 </addcategory>

 <getcategories>
 </getcategories>
 
</response>
\end{verbatim}

\paragraph{Return Codes}

Die Return-Codes stehen immer in den Tags $<$return-value$>$xx$<$/return-value$>$. Die Description ist optional.

\begin{description}
\item[0] Alles ok 

\item[1] Fehler, konnte URL nicht finden 

\item[2] Ung�ltige Anfrage (Pflicht-Parameter fehlen oder haben die L�nge null) 

\item[3] Inkompatible Interface-Version (die Anfrage hat eine Interface-Version benutzt die der Server nicht unterst�tzt) 

\item[4] Kategorie existiert schon (beim hinzufuegen einer neuen Kategorie) 

\item[5] Kategorie nicht gefunden (beim suchen nach einer bestimmten Kategorie) 

\item[-1] Anderer Fehler 
\end{description}

Die Return-Descriptions sind frei w�hlbar, sollten aber dem Return-Code semantisch entsprechen ;-)


\subsection{Seite hinzuf�gen}
\subsubsection{Anfrage an Server}

\begin{description}
\item[command=addpage] Der Kommando-Name 

\item[addpage.pl] Der (default) Skriptname der die Anfrage entgegennimmt (dann f�llt das Argument command weg). 

\item[category=453] Die Kategorien-Nummer zu der der Link hinzugefuegt werden soll, in dieser Version nur genau eine oder keine. Wird keine Kategorie angegeben, so soll die Top-Kategorie genommen werden (also nicht gefunden werden wenn bei der Suche eine genauere Kategorie angegeben wird). 

\item[url=http://yy.org/blabla.html] Die zu indizierende url, das Protokoll ist mit anzugeben 
\end{description}

\subsubsection{XML Antwort}
xml oder html um �ber Erfolg oder Misserfolg zu berichten

Verwendet in der Antwort das XML-Tag: $<$addpage$>$
\begin{verbatim}
<response>

 <interface-version>2</interface-version>

 <return-value>0</return-value>
 <return-description>OK</return-description>

 <server>
  <name>TeamFound</name>
  <version>0.2</version>
 </server>

 <addpage>
  <url>http://blabla.html</url>
 </addpage>
 
</response>
\end{verbatim}

\subsubsection{HTML Antwort}
siehe Allgemeines zu HTML-Antworten


\subsection{Suchen}
\subsubsection{Anfrage an Server}
\begin{description}
\item[command=search] Der Kommando-Name 

\item[search.pl] Der (default) Skriptname der die Anfrage entgegennimmt (dann f�llt das Argument command weg). 
\item[keyword=zzz] Eine durch ' ' (space) getrennte Liste der Suchw�rter die alle vorkommen sollen (URL-Codiert mit ''\%20'' getrennt, also eine Suche nach auto und tv wird als keyword=auto\%20tv codiert). Dies beinhaltet eine serverseitig impliziete UND-Verkn�pfung aller Suchworte. Fehlt der Parameter so soll ein Fehler zur�ckgegeben werden. Die genaue Definition einer umfassenderen Suchanfragen-Sprache (wie OR, NOT etc.) ist noch festzulegen, wird aber auf spaetere Interface-Milestones verschoben. 

\item[category=5\&category=9\&category=15] Eine Liste aller Kategorien-IDs, jeweils immer wieder neu mit category= durch die die Suche gemacht werden soll. Fehlt der Paramater, so soll alles durchsucht werden. (Grund: HTML-Formulare machen dies bei Listen und ComboBoxen mit Mehrfachauswahl genau so). 
\end{description}

\subsubsection{XML Antwort}
Verwendet in der Antwort das XML-Tag: $<$search$>$

\begin{verbatim}
<response>

 <interface-version>2</interface-version>

 <return-value>0</return-value>
 <return-description>OK</return-description>

 <server>
  <name>TeamFound</name>
  <version>0.2</version>
 </server>

 <search>

  <keywords>
   <word>xxx</word>
   <word>yyy</word>
  </keywords>

  <result>

   <count>30</count>
   <offset>0</offset>

   <found>
    <url>http://xxx.html</url>
    <title>Der Titel der Seite</title>
    <incategory>5</incategory>
   </found>

   <found>
    <url>http://xxx.html</url>
    <title>Der Titel der Seite</title>
    <incategory>5</incategory>
   </found>

   <found>
    <url>http://xxx.html</url>
    <title>Der Titel der Seite</title>
    <incategory>5</incategory>
    <incategory>3</incategory>
    <incategory>7</incategory>
   </found>

  </result>
 </search>
 
</response>
\end{verbatim}
\begin{itemize}
\item Die Keywords sind zur Fehlerbehandlung alle nochmals mit anzugeben. 

\item Jeder gefundene Link wird in einem eingenen $<$found$>$ Block augegeben. 

\item Wurde kein Suchergebnis gefunden, so ist in $<$count$>$ die Anzahl 0 einzutragen, der Return-Code aber immernoch 0 (OK) wenn sonst alles ok war 
\end{itemize}

\subsubsection{HTML Antwort}
HTML-Seite mit URLs gefundener �bereinstimmungen

\subsection{Kategorien von Server abfragen}
\subsubsection{Anfrage an Server}

\begin{description}
\item[command=getcategories] Der Kommando-Name 
\item[getcategories.pl] Der (default) Skriptname der die Anfrage entgegennimmt (dann f�llt das Argument command weg). 
\item[projectID=5] Nummer des projekts f�r diese Anfrage 
\end{description}

\subsubsection{XML Antwort}
Verwendet in der Antwort das XML-Tag: $<$getcategories$>$
\begin{verbatim}
<response>

 <interface-version>2</interface-version>

 <return-value>0</return-value>
 <return-description>OK</return-description>

 <server>
  <name>TeamFound</name>
  <version>0.2</version>
 </server>

 <getcategories>
  <category>
   <name>name der kategorie</name>
   <description>laengere beschreibung</description>
   <id>0</id>
   <subcategories>
    <category>
     ...
    </category>
    <category>
     ...
    </category>
   </subcategories>
  </category>
 </getcategories>
 
</response>
\end{verbatim}

Hinweis: die Top-Kategorie hat immer die ID 0 (null) 

\subsubsection{HTML Antwort}
siehe Allgemeines zu HTML-Antworten

\subsection{Kategorie hinzuf�gen}
\subsubsection{Anfrage an Server}
\begin{description}
\item[command=addcategory] Der Kommando-Name 

\item[addcategory.pl] Der (default) Skriptname der die Anfrage entgegennimmt (dann f�llt das Argument command weg). 

\item[name=kategoriename] Den Namen den die neue Kategorie bekommen soll 

\item[subcategoryof=45] Die Kategorie der die neue Kategorie untergeordnet werden soll. Soll eine neue 1st-Level Kategorie erstellt werden ist hier 0 (null) anzugeben. 

\item[description] Eine etwas l�ngere Beschreibung der Kategorie (max. 255 Zeichen) 
\end{description}

\subsubsection{XML Antwort}
Verwendet in der Antwort das XML-Tag: $<$addcategory$>$
\begin{verbatim}
<response>

 <interface-version>2</interface-version>

 <return-value>0</return-value>
 <return-description>OK</return-description>

 <server>
  <name>TeamFound</name>
  <version>0.2</version>
 </server>

 <addcategory>
  <name>kategoriename</name>
  <gotid>53</gotid>
 </addcategory>
 
</response>
\end{verbatim}
\begin{itemize}
\item der neue Name
\item die ID die die neue Kategorie bekommen hat
\item name und gotid fallen weg wenn die kategorie schon existiert (return-value 4 - Kategorie existiert schon) oder die subcategoryof-id nicht existiert (return-value 5 - Kategorie nicht gefunden). 
\end{itemize}

\subsubsection{HTML Antwort}
siehe Allgemeine HTML-Antwort

\subsection{Alle Projekte auslesen}
\subsubsection{Anfrage an Server}
\begin{description}
\item[command=getprojects] Der Kommando-Name 
\item[addcategory.pl] Der (default) Skriptname der die Anfrage entgegennimmt (dann f�llt das Argument command weg). 
\end{description}

\subsubsection{XML Antwort}
Verwendet in der Antwort das XML-Tag: $<$projects$>$
\begin{verbatim}
<response>

 <interface-version>2</interface-version>

 <return-value>0</return-value>
 <return-description>OK</return-description>

 <server>
  <name>TeamFound</name>
  <version>0.2</version>
 </server>

 <projects>
   <project>
     <name>prjectname</name>
     <description>beschreibung</description>
     <id>7</id>
   </project>
   ...
 </project>
 
</response>
\end{verbatim}

\subsubsection{HTML Anwort}
siehe Allgemeine HTML-Antwort

\subsection{L�schen einer Kategorie}
Wird erst in Interface Milestone 3 implementiert werden. Es stellt sich die Frage was mit den in dieser Kategorie bereits befindlichen Links gemacht wird .. verschieben?

\chapter{Changelog}
\section{Firefox Toolbar}
\label{changelog}
Das originale \index{Changelog}\footnote{\texttt{http://teamfound.berlios.de/\#firefoxchangelog0.7}} (in englischer Sprache):
\subsection{Firefox \& Flock toolbar 0.7 (07-DEC-2005)}
\begin{itemize}
\item Bug-Fix: Add-Button actually works now
\item Label with current URL removed (though we are not going to replace the location-bar)
\item Compatibility with Flock browser tested and added to install.rdf
\end{itemize}

\subsection{Firefox toolbar 0.6 (30-NOV-2005)}

\begin{itemize}
\item Input-field now behaves the same like the normal Firefox locationbar (including history popup) when urls are typed. Soon this can replace the Firefox locationbar and the Firefox searchbar with only one input-field ;-)
\item Settings-Dialog allows to give custom external search engines
\item Settings-Dialog allows to only search external, only search TeamFound or search both at once
\end{itemize}

\subsection{Firefox toolbar 0.5 (28-NOV-2005)}

\begin{itemize}
\item Google search results are actually working now, including further results
\item Compatible with Firefox 1.5 release candidates
\end{itemize}

\subsection{Firefox toolbar 0.4 (27-NOV-2005)}

\begin{itemize}
\item Clicking on TF-Icon opens preferences dialog, here you can setup your TeamFound server url and choose between two search result layouts
\item Add page - Adds the current page to the TeamFound-Index.
\item text field - Here you can enter search words or urls. The toolbar tries to automatically figure out what you meant and though will either search or just visit the given url.
\item current url - The current url is always displayed at the end of the toolbar, though not overwriting your current search words.
\end{itemize}

\subsection{Firefox toolbar 0.3 (24-NOV-2005)}

\begin{itemize}
\item adds search at extern search-engine
\item shows google and teamfound search-results in same window
\end{itemize}

\subsection{Firefox toolbar 0.2 (22-NOV-2005)}

\begin{itemize}
\item first release at berlios.de
\item available as .xpi
\end{itemize}

\subsection{Firefox toolbar 0.1 (19-NOV-2005)}

\begin{itemize}
\item initial release
\end{itemize}



\begin{thebibliography}{keine ah}
\addcontentsline{toc}{chapter}{Literaturverzeichnis}

\bibitem[Apa02]{} \emph{Apache Ant 1.5.1 Manual}. Apache Software Foundation, http://ant.apache.org/manual/, 2002.
\bibitem[Dat00]{} C. J. Date. \emph{An Introduction to Database Systems}. Addison Wesley, 2000.
\bibitem[ExFF]{} \emph{Extend Firefox}. Mozilla, http://developer.mozilla.org/mozilla-org/contests/extendfirefox/documentation.php
\bibitem[FFTT]{} \emph{Firefox Toolbar Tutorial}. Born Geek, http://www.borngeek.com/firefox/toolbar-tutorial/
\bibitem[Fri97]{} Jeffrey E. F. Friedl. \emph{Mastering Regular Expressions}. O'Reilly, 1997.
\bibitem[Gswed]{} \emph{Getting started with extension development}. mozillaZine, http://kb.mozillazine.org/Getting\_started\_with\_extension\_development.
\bibitem[Goo00]{} M. Goossens, F. Mittelbach und A. Samarin. \emph{Der \LaTeX \ Begleiter}. Addison Wesley, 2000.
\bibitem[LucQS]{} \emph{Apache Lucene Query Syntax}. Apache Software Foundation http://lucene.apache.org/java/docs/queryparsersyntax.html .
\bibitem[Oua02]{} S. Oualline. \emph{Vi IMproved -- Vim}. New Riders, 2002 .
\bibitem[PolPos]{} \emph{PolePosition}. the open source database benchmark http://www.polepos.org.
\bibitem[Roac04]{} Eric. \emph{How to create Firefox extensions}. roachfiend.com, http://roachfiend.com/archives/2004/12/08/how-to-create-firefox-extensions
\bibitem[Spo01]{} J. Spolsky. \emph{User Interface Desing for Programmers}. Apress, 2001 .
\bibitem[XULPL]{} \emph{XUL Planet} XUL Planet, http://xulplanet.com .

\end{thebibliography}

