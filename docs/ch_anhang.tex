\chapter{Interface}
\section{Interface Milestone 2}
\label{interface2}
\subsection{Allgemein}
\subsubsection{Tests}
Zum Testen der Serverseitigen Interface-Implementation existiert folgende Webseite, die entsprechende Requests erstellen kann: 

\texttt{http://teamfound.berlios.de/test\_the\_server\_interface\_m2.html}
\\
\\
Zum Testen der Clientseitigen Interface-Implementation existiert folgendes Perl-Skript, welches den Parameter command auswertet und eine enstsprechende statische XML-Antwort zurueckgibt: 
\\
\texttt{http://teamfound.dyndns.org/interface-m2/testmilestone2.pl}
\\
Demo: 
\\
\texttt{http://teamfound.dyndns.org/interface-m2/testmilestone2.pl?command=getcategories}
\\
Das Skript liegt auch im SVN unter \texttt{teamfound/interface/milestone2/testmilestone2.pl}

\subsubsection{Alle Anfragen an den Server}

Alle Anfrage-Parameter werden in Form von HTTP GET oder POST Variablen �bertragen. Prinzipiell gibt es eine Unterscheidung, ob f"ur jedes Kommando eine eigene URL verwendet wird, oder ob das Kommando in Form einer weiteren HTTP-GET Variablen mit "ubertragen wird. Unsere Implementation des  Servers verwendet den zus"atzlichen HTTP-GET Parameter \texttt{command}. In diesem Interface haben wir aber beide M"oglichkeiten spezifiziert.

\begin{description}
\item[want=xml or html] Die zu erwartende Antwort soll in xml oder html-format sein (default soll html werden, da dann einfache Link-Clients m�glich sind) 

\item[version=2] Milestone-Version des Interfaces 

\item[command=search] Das Kommando das ausgef�hrt werden soll. Dieses Argument kann wegfallen wenn f�r jedes Kommando eine eigene, gleichnamige Anfrage-URL existiert. 
\end{description}

\subsubsection{HTML-Antwort}
Die Clients geben an ob Sie eine HTML oder eine XML Antwort erwarten.

Soll eine vollst�ndige HTML-Seite zur�ckgeben, die direkt im Browser angezeigt werden kann. Die Return-Values wie bei einer XML-Antwort m�ssen in diesem Fall nicht zur�ckgegeben werden, sondern der Text sollte gleich eine entsprechende Meldung beinhalten.

\subsubsection{XML-Antwort}
\begin{description}
\item[$<$response$>$] Umschliesst alle anderen xml-tags und gibt die XSD-Datei zum verifizieren der XML-Daten an 
\item[$<$interface-version$>$] Gibt immer die Interface-Version an, in der die Antwort des Servers formuliert wurde 
\item[$<$server$>$] Gibt Name und Versionsnummer des Servers an. Der Server der unter der URL https://developer.berlios.de/projects/teamfound entwickelt wird gibt den Namen "TeamFound" zur�ck. Clones d�rfen Ihren eigenen Namen nat�rlich frei w�hlen. 
\item[$<$addpage$>$ $<$search$>$ $<$addcategory$>$ $<$getcategories$>$] Diese Tags beinhalten die Antworten auf die gleichnamigen Anfragen an den Server. Pro Anfrage darf nur eines dieser Tags vorkommen. 
\item[$<$return-value$>$ $<$return-description$>$] Der Return-Value bzw. Fehler-Code falls etwas schiefging. Die Description wird nicht genauer spezifiziert, sollte aber semantisch mit dem aufgetretenen Fehler �bereinstimmen. 
\item[$<$project-counter$>$] Bei jeder �nderung des Kategorien-Baumes( soll der Server einen internen Z�hler um eins inkrementieren. Der aktuelle Wert soll bei jeder XML-Antwort mit �bertragen werden. (Dann weiss die Toolbar wann sie selber die Kategorien neu vom Server abfragen muss.)Jedes Projekt fuehrt einen Counter, somit muss der Client nur die fuer ihn intressanten B"aume beachten. 
\end{description}

\begin{verbatim}
<response xmlns:xsi="http://www.w3.org/2001/XMLSchema-instance"
xsi:noNamespaceSchemaLocation="teamfound-interface-milestone2.xsd">

 <interface-version>2</interface-version>

 <return-value>0</return-value>
 <return-description>OK</return-description>

 <project-counter>
   <project>
     <projectID>0</projectID>
     <count>3</count>
   </project>
   <project>
     <projectID>3</projectID>
     <count>5</count>
   </project>
 </project-counter>

 <category-counter>54</category-counter>

 <server>
  <name>TeamFound</name>
  <version>0.2</version>
 </server>

 <addpage>
 </addpage>

 <search>
 </search>

 <addcategory>
 </addcategory>

 <getcategories>
 </getcategories>
 
</response>
\end{verbatim}

\paragraph{Return Codes}

Die Return-Codes stehen immer in den Tags $<$return-value$>$xx$<$/return-value$>$. Die Description ist optional.

\begin{description}
\item[0] Alles ok 

\item[1] Fehler, konnte URL nicht finden 

\item[2] Ung�ltige Anfrage (Pflicht-Parameter fehlen oder haben die L�nge null) 

\item[3] Inkompatible Interface-Version (die Anfrage hat eine Interface-Version benutzt die der Server nicht unterst�tzt) 

\item[4] Kategorie existiert schon (beim hinzufuegen einer neuen Kategorie) 

\item[5] Kategorie nicht gefunden (beim suchen nach einer bestimmten Kategorie) 

\item[-1] Anderer Fehler 
\end{description}

Die Return-Descriptions sind frei w�hlbar, sollten aber dem Return-Code semantisch entsprechen ;-)


\subsection{Seite hinzuf�gen}
\subsubsection{Anfrage an Server}

\begin{description}
\item[command=addpage] Der Kommando-Name 

\item[addpage.pl] Der (default) Skriptname der die Anfrage entgegennimmt (dann f�llt das Argument command weg). 

\item[category=453] Die Kategorien-Nummer zu der der Link hinzugefuegt werden soll, in dieser Version nur genau eine oder keine. Wird keine Kategorie angegeben, so soll die Top-Kategorie genommen werden (also nicht gefunden werden wenn bei der Suche eine genauere Kategorie angegeben wird). 

\item[url=http://yy.org/blabla.html] Die zu indizierende url, das Protokoll ist mit anzugeben 
\end{description}

\subsubsection{XML Antwort}
xml oder html um �ber Erfolg oder Misserfolg zu berichten

Verwendet in der Antwort das XML-Tag: $<$addpage$>$
\begin{verbatim}
<response>

 <interface-version>2</interface-version>

 <return-value>0</return-value>
 <return-description>OK</return-description>

 <server>
  <name>TeamFound</name>
  <version>0.2</version>
 </server>

 <addpage>
  <url>http://blabla.html</url>
 </addpage>
 
</response>
\end{verbatim}

\subsubsection{HTML Antwort}
siehe Allgemeines zu HTML-Antworten


\subsection{Suchen}
\subsubsection{Anfrage an Server}
\begin{description}
\item[command=search] Der Kommando-Name 

\item[search.pl] Der (default) Skriptname der die Anfrage entgegennimmt (dann f�llt das Argument command weg). 
\item[keyword=zzz] Eine durch ' ' (space) getrennte Liste der Suchw�rter die alle vorkommen sollen (URL-Codiert mit ''\%20'' getrennt, also eine Suche nach auto und tv wird als keyword=auto\%20tv codiert). Dies beinhaltet eine serverseitig impliziete UND-Verkn�pfung aller Suchworte. Fehlt der Parameter so soll ein Fehler zur�ckgegeben werden. Die genaue Definition einer umfassenderen Suchanfragen-Sprache (wie OR, NOT etc.) ist noch festzulegen, wird aber auf spaetere Interface-Milestones verschoben. 

\item[category=5\&category=9\&category=15] Eine Liste aller Kategorien-IDs, jeweils immer wieder neu mit category= durch die die Suche gemacht werden soll. Fehlt der Paramater, so soll alles durchsucht werden. (Grund: HTML-Formulare machen dies bei Listen und ComboBoxen mit Mehrfachauswahl genau so). 
\end{description}

\subsubsection{XML Antwort}
Verwendet in der Antwort das XML-Tag: $<$search$>$

\begin{verbatim}
<response>

 <interface-version>2</interface-version>

 <return-value>0</return-value>
 <return-description>OK</return-description>

 <server>
  <name>TeamFound</name>
  <version>0.2</version>
 </server>

 <search>

  <keywords>
   <word>xxx</word>
   <word>yyy</word>
  </keywords>

  <result>

   <count>30</count>
   <offset>0</offset>

   <found>
    <url>http://xxx.html</url>
    <title>Der Titel der Seite</title>
    <incategory>5</incategory>
   </found>

   <found>
    <url>http://xxx.html</url>
    <title>Der Titel der Seite</title>
    <incategory>5</incategory>
   </found>

   <found>
    <url>http://xxx.html</url>
    <title>Der Titel der Seite</title>
    <incategory>5</incategory>
    <incategory>3</incategory>
    <incategory>7</incategory>
   </found>

  </result>
 </search>
 
</response>
\end{verbatim}
\begin{itemize}
\item Die Keywords sind zur Fehlerbehandlung alle nochmals mit anzugeben. 

\item Jeder gefundene Link wird in einem eingenen $<$found$>$ Block augegeben. 

\item Wurde kein Suchergebnis gefunden, so ist in $<$count$>$ die Anzahl 0 einzutragen, der Return-Code aber immernoch 0 (OK) wenn sonst alles ok war 
\end{itemize}

\subsubsection{HTML Antwort}
HTML-Seite mit URLs gefundener �bereinstimmungen

\subsection{Kategorien von Server abfragen}
\subsubsection{Anfrage an Server}

\begin{description}
\item[command=getcategories] Der Kommando-Name 
\item[getcategories.pl] Der (default) Skriptname der die Anfrage entgegennimmt (dann f�llt das Argument command weg). 
\item[projectID=5] Nummer des projekts f�r diese Anfrage 
\end{description}

\subsubsection{XML Antwort}
Verwendet in der Antwort das XML-Tag: $<$getcategories$>$
\begin{verbatim}
<response>

 <interface-version>2</interface-version>

 <return-value>0</return-value>
 <return-description>OK</return-description>

 <server>
  <name>TeamFound</name>
  <version>0.2</version>
 </server>

 <getcategories>
  <category>
   <name>name der kategorie</name>
   <description>laengere beschreibung</description>
   <id>0</id>
   <subcategories>
    <category>
     ...
    </category>
    <category>
     ...
    </category>
   </subcategories>
  </category>
 </getcategories>
 
</response>
\end{verbatim}

Hinweis: die Top-Kategorie hat immer die ID 0 (null) 

\subsubsection{HTML Antwort}
siehe Allgemeines zu HTML-Antworten

\subsection{Kategorie hinzuf�gen}
\subsubsection{Anfrage an Server}
\begin{description}
\item[command=addcategory] Der Kommando-Name 

\item[addcategory.pl] Der (default) Skriptname der die Anfrage entgegennimmt (dann f�llt das Argument command weg). 

\item[name=kategoriename] Den Namen den die neue Kategorie bekommen soll 

\item[subcategoryof=45] Die Kategorie der die neue Kategorie untergeordnet werden soll. Soll eine neue 1st-Level Kategorie erstellt werden ist hier 0 (null) anzugeben. 

\item[description] Eine etwas l�ngere Beschreibung der Kategorie (max. 255 Zeichen) 
\end{description}

\subsubsection{XML Antwort}
Verwendet in der Antwort das XML-Tag: $<$addcategory$>$
\begin{verbatim}
<response>

 <interface-version>2</interface-version>

 <return-value>0</return-value>
 <return-description>OK</return-description>

 <server>
  <name>TeamFound</name>
  <version>0.2</version>
 </server>

 <addcategory>
  <name>kategoriename</name>
  <gotid>53</gotid>
 </addcategory>
 
</response>
\end{verbatim}
\begin{itemize}
\item der neue Name
\item die ID die die neue Kategorie bekommen hat
\item name und gotid fallen weg wenn die kategorie schon existiert (return-value 4 - Kategorie existiert schon) oder die subcategoryof-id nicht existiert (return-value 5 - Kategorie nicht gefunden). 
\end{itemize}

\subsubsection{HTML Antwort}
siehe Allgemeine HTML-Antwort

\subsection{Alle Projekte auslesen}
\subsubsection{Anfrage an Server}
\begin{description}
\item[command=getprojects] Der Kommando-Name 
\item[addcategory.pl] Der (default) Skriptname der die Anfrage entgegennimmt (dann f�llt das Argument command weg). 
\end{description}

\subsubsection{XML Antwort}
Verwendet in der Antwort das XML-Tag: $<$projects$>$
\begin{verbatim}
<response>

 <interface-version>2</interface-version>

 <return-value>0</return-value>
 <return-description>OK</return-description>

 <server>
  <name>TeamFound</name>
  <version>0.2</version>
 </server>

 <projects>
   <project>
     <name>prjectname</name>
     <description>beschreibung</description>
     <id>7</id>
   </project>
   ...
 </project>
 
</response>
\end{verbatim}

\subsubsection{HTML Anwort}
siehe Allgemeine HTML-Antwort

\subsection{L�schen einer Kategorie}
Wird erst in Interface Milestone 3 implementiert werden. Es stellt sich die Frage was mit den in dieser Kategorie bereits befindlichen Links gemacht wird .. verschieben?

\chapter{Logbuch}
\section{Projekt Logbuch}
Dies ist ein Snapshot des Projekt-Logbuchs vom 17. April 2006. Die jeweils aktuelle Version kann unter \texttt{http://wiki.jonasheese.de/index.php/Logbuch} eingesehen werden.
\subsection{2006}

\subsubsection{April 2006}
\paragraph{17. April 2006}
\begin{itemize}
\item Firefox-Extension v0.8 released Kechel 20:07, 17. Apr 2006 (CEST) 
\end{itemize}

\paragraph{11. April 2006}
\begin{itemize}
\item Vortrag gehalten Kechel 21:46, 11. Apr 2006 (CEST) 
\end{itemize}

\paragraph{10. April 2006}
\begin{itemize}
\item Vortrag-Slides fertig Kechel 20:14, 10. Apr 2006 (CEST)
\item FF-Extension 0.8 beta fertig incl. default-Kategorie (reicht fuer Praesentation ;-) Kechel 20:14, 10. Apr 2006 (CEST) 
\end{itemize}

\paragraph{9. April 2006}
\begin{itemize}
\item Vortrag-Slides angefangen Kechel 20:14, 10. Apr 2006 (CEST)
\item FF-Extension 0.8 funktioniert schon ganz gut Kechel 20:14, 10. Apr 2006 (CEST) 
\end{itemize}

\paragraph{8. April 2006}
\begin{itemize}
\item Ausarbeitung praktisch fertig incl. Anhang, Lizenz und Index Kechel 20:14, 10. Apr 2006 (CEST) 
\end{itemize}

\paragraph{7. April 2006}
\begin{itemize}
\item Ausarbeitung Anhang Kechel 20:14, 10. Apr 2006 (CEST)
\item ff-toolbar kategorien von echtem server geht Kechel 20:14, 10. Apr 2006 (CEST) 
\end{itemize}

\paragraph{4. April 2006}

\begin{itemize}
\item Ausarbeitung weitergeschrieben, Interface und Protokoll Kechel 20:14, 10. Apr 2006 (CEST)
\item FF-Extension 0.8 weiterprogrammiert Kechel 20:14, 10. Apr 2006 (CEST) 
\end{itemize}

\paragraph{2. April 2006}

\begin{itemize}
\item Ausarbeitung angefangen Kechel 20:14, 10. Apr 2006 (CEST) 
\end{itemize}


\subsubsection{M"arz 2006}

\paragraph{31. Maerz 2006}
\begin{itemize}
\item FF Client kann kategorien, kategorien durchsuchen und auch seiten zu kategorien hinzufuegen, die Anzeige von Suchergebnissen laesst noch sehr zu wuenschen uebrig Kechel 20:14, 10. Apr 2006 (CEST) 
\end{itemize}

\paragraph{12. Maerz 2006}
\begin{itemize}
\item XML-Kategorien-Baum wird in FF-Extension korrekt angezeigt. Kechel 16:07, 12. M�r 2006 (CET) 
\end{itemize}

\subsubsection{Januar 2006}
\paragraph{22. Januar 2006}
\begin{itemize}
\item Diese Woche den DBLayer vorlaeufig beendet. Moddin 14:27, 22. Jan 2006 (CET)
\item DBerlaeuterung im Wiki. Moddin 14:27, 22. Jan 2006 (CET)
\item Vortrag im Wiki vorbereitet, Themenauflistung, konkrete Vorschlaege. Kechel 21:51, 22. Jan 2006 (CET) 
\end{itemize}

\paragraph{12. Januar 2006}

\begin{itemize}
\item FF-Toolbar erstellt ersten M2 request (getcategories), werted die XML-Antwort aus und zeigt die erste Kategorie als Menuepunkt an. Kechel 17:05, 12. Jan 2006 (CET)
\item Testscript fuer Clients Milestone 2 geschrieben (http://wiki.jonasheese.de/index.php/Interface\_Milestone\_2\#Tests) Kechel 13:55, 12. Jan 2006 (CET) 
\end{itemize}

\subsection{2005}
\subsubsection{Dezember 2005}
\paragraph{7. Dezember 2005}
\begin{itemize}
\item FF-Toolbar 0.7 released (bug-fix). Kechel 16:08, 7. Dez 2005 (CET)
\item Test-Seite fuer Server Interface Milestone 2 Implementation erstellt (http://teamfound.berlios.de/test\_the\_server\_interface\_m2.html). Kechel 20:09, 7. Dez 2005 (CET) 
\end{itemize}

\paragraph{6. Dezember 2005}
\begin{itemize}
\item Label mit aktueller URL aus FF-Toolbar entfernt. Kechel 18:48, 6. Dez 2005 (CET) 
\end{itemize}

\paragraph{5. Dezember 2005}
\begin{itemize}
\item Code Review von Martins Zeug, morgen m�ssen wir auf jedenfall die Architektur genauer festelegen, das wir kein Chaos produzieren. --Jonas 21:18, 5. Dez 2005 (CET) 
\end{itemize}

\paragraph{2. Dezember 2005}

\begin{itemize}
\item Toolbar jetzt auch ueber http://addons.mozilla.org verfuegbar. Kechel 19:22, 2. Dez 2005 (CET) 
\end{itemize}
\begin{verbatim}
TeamFound 0.6 - Approval Granted
Your item, TeamFound 0.6, has been reviewed by a 
Mozilla Update editor who took the following action:
Approval Granted

Please Note: It may take up to 30 minutes for your 
extension to be available for download.

Your item was tested by Chris Blore using Firefox 1.5 on Windows XP.
Editor's Comments:
Thanks for submitting
\end{verbatim}

\paragraph{1. Dezember 2005}
\begin{itemize}
\item Firefox-Toolbar ist jetzt auch eine Flock-Toolbar (also compatibel zum flock-browser http://www.flock.com). Kechel 19:13, 1. Dez 2005 (CET) 
\item Firefox-Toolbar ist jetzt auch eine Flock-Toolbar (also compatibel zum flock-browser http://www.flock.com). Kechel 19:13, 1. Dez 2005 (CET) 
\end{itemize}

\subsubsection{November 2005}
\paragraph{30. November 2005}

\begin{itemize}
\item Milestone 1 Version als Servlet gebastelt.Moddin 19:57, 30. Nov 2005 (CET)
\item Firefox toolbar 0.6 released. Kechel 23:54, 30. Nov 2005 (CET) 
\end{itemize}


\paragraph{29. November 2005}

\begin{itemize}
\item Gemeinsam Interface Milestone 2 verabschiedet. Kechel 15:23, 30. Nov 2005 (CET) 
\end{itemize}

\paragraph{28. November 2005}
\begin{itemize}
\item Firefox-Toolbar 0.5 released, google links funktionieren endlich (auch die weitere-ergebnisse-links). In CustomizeGoogle Malingliste gefragt wie wir unsere extensions zur Kooperation bringen koennten. Kechel 16:15, 28. Nov 2005 (CET) 
\end{itemize}

\paragraph{27. November 2005}

\begin{itemize}
\item Toolbar bei addons.mozilla.org hochgeladen, waiting for review and aproval. Kechel 23:16, 27. Nov 2005 (CET)
\item Firefox-Toolbar v0.4 ist fertig, asyncrones laden der TeamFound und google ergebnisse, sowie eingegebene urls werden nicht gesucht sondern direkt angesprungen. Kechel 15:08, 27. Nov 2005 (CET) 
\end{itemize}

\paragraph{26. November 2005}

\begin{itemize}
\item Firefox-Toolbar hat nun auch einen Einstellungen-Dialog fuer die Server-Adresse. Kechel 22:49, 26. Nov 2005 (CET) 
\end{itemize}

\paragraph{25. November 2005}
\begin{itemize}
\item XSD-Datei zum verifizieren der XLM-Daten nach Interface Milestone 2 erzeugt. Kechel 18:12, 25. Nov 2005 (CET) 
\end{itemize}

\paragraph{24. November 2005}

\begin{itemize}
\item Firefox-Toolbar sucht jetzt auch selber noch bei google und zeigt die Ergebnisse von TeamFound und google nebeneinander an. Version 0.3 der Toolbar released. Kechel 22:54, 24. Nov 2005 (CET) 
\end{itemize}

\paragraph{23. November 2005}

\begin{itemize}
\item Am ServerProgramm ein paar Verbesserungen vorgenommen hinsichtlich der Performance
\item Script und java so erweitert das adding bei gesetztem lock wartet(sehr simple variante) Moddin 21:17, 23. Nov 2005 (CET)
\item Server-Release 0.1 auf Berlios geladen, kleine install.txt hinzugefuegt, $<$category-counter$>$ zu Interface Milestone 2 hinzugefuegt. Kechel 16:23, 23. Nov 2005 (CET)
\item Web-Client geschrieben (einfaches HTML-Formular zum suchen, und link zum adden neuer seiten). Kechel 20:21, 23. Nov 2005 (CET) 
\end{itemize}

\paragraph{22. November 2005}
\begin{itemize}
\item Firefox-Extension 0.2 xpi-file erstellt und als erstes file auf berlios released, installations-anleitung angepasst und link auf end-user homepage gesetzt. Kechel 19:46, 22. Nov 2005 (CET)
\item Interface Milestone 2 spezifiziert (als Vorschlag, aber schon sehr weit ausgearbeitet!) Kechel 21:59, 22. Nov 2005 (CET) 
\end{itemize}

\paragraph{21. November 2005}

\begin{itemize}
\item An den Server Interna gearbeitet um Milestone 1. naeherzukommen (Files sind im svn, fortlaufenden Index von Seiten kann gebaut und durchsucht werden) Moddin 21:05, 21. Nov 2005 (CET) 
\end{itemize}

\paragraph{19. November 2005}

\begin{itemize}
\item Firefox-Toolbar ist in version 0.1 fertig und einsatzbereit. Es koennen Seiten hinzugefuegt werden und diese wieder durchsucht werden. Der Server funktioniert ebenfalls in einer ersten Version bei uns lokal auf einem Rechner. Der Index wird jedoch bei jeder neu hinzugefuegten seite komplett ueberschrieben, so dass immer nur die zuletzt hinzugefuege Seite durchsucht wird ;-) Kechel 20:13, 19. Nov 2005 (CET)
\item Installations-Anleitung fuer Server-Milestone 1 geschrieben. Kechel 20:39, 19. Nov 2005 (CET)
\item Installations-Anleitung und kleine Doku fuer Firefox-Toolbar Milestone 1 erstellt. Kechel 22:48, 19. Nov 2005 (CET)
\item Logo auf http://teamfound.berlios.de upgeloaded und in Pr�sentation vorgestellt. Kechel 23:07, 19. Nov 2005 (CET)
\item Icon in Firefox-Toolbar eingebaut und screenshot ins wiki getan damit endlich mal etwas optisch anspruchsvolles im wiki ist ;-) Kechel 00:41, 20. Nov 2005 (CET) 
\end{itemize}

\paragraph{17. November 2005}
\begin{itemize}
\item Erste Server-Perl Skripte in SVN getan und in Server entsprechende Beschreibung eingefuegt. Kechel 15:59, 17. Nov 2005 (CET) 
\end{itemize}

\paragraph{15. November 2005}
\begin{itemize}
\item Interface-Spezifikation f�r Milestone 1 fertig. Kechel 11:03, 15. Nov 2005 (CET)
\item Erste Server-Perl Scripte erstellt. Jan \& Moddin 
\end{itemize}

\paragraph{12. November 2005}
\begin{itemize}
\item kleiner Testlauf mit Lucene-Demo bei mir ... 270 MB hat ungefaehr 3 min gebraucht zum index erstellen und nahm 14 MB ein. Ausserdem haben sie auch nen HTMLParser in der Demo aber nur fuer reine HTML-Files ... alles andere wird dann nicht mit index versehen, aber immerhin fuer uns schon sehr nuetzlich. Moddin 15:04, 12. Nov 2005 (CET) 
\end{itemize}

\paragraph{09. November 2005}

\begin{itemize}
\item Mailingliste teamfound-development@lists.berlios.de eingerichtet und entsprechende links auf diesem wiki plaziert. Kechel 15:25, 9. Nov 2005 (CET)
\item Erster SVN-Checkout sowie erste Firefox-Toolbar upgeloaded. Kechel 16:06, 9. Nov 2005 (CET) 
\end{itemize}

\paragraph{08. November 2005}
\begin{itemize}
\item Erste eigene Firefox-Extension erstellt die auch funktioniert (Menu-Item 'TeamFound' der beim anklicken eine MsgBox ausgibt ;-) Kechel 23:25, 8. Nov 2005 (CET) 
\end{itemize}

\paragraph{07. November 2005}
\begin{itemize}
\item Nach nutzbaren tools fuer den Server gesucht und alles was ich so gefunden habe ins Wiki unter Server geschrieben.Moddin 16:43, 7. Nov 2005 (CET)
\item Nach Informationen zum IE Plugin gesucht und alles unter Toolbar als Extension f�r Internet Explorer eingetragen.Andreas 22:34, 7. Nov 2005 (CET)
\item Erste Firefox-Extension versucht zu programmieren .. naja, und zum ersten mal von XUL gehoert ;-) Aber dafuer hat die HelloWorld-Extension gleich funktioniert! Kechel 23:45, 7. Nov 2005 (CET) 
\end{itemize}

\paragraph{03. November 2005}

\begin{itemize}
\item berlios-Services sind inzwischen aktiv, also shell login geht, die domain teamfound.berlios.de ist erreichbar und ich habe auch gleich mal einen minimalen content aufgesetzt. Kechel 14:01, 3. Nov 2005 (CET) 
\item Projekt bei berlios.de wurde approved. Projekt ein bisschen eingerichtet, wichtige infos auf wiki uebertragen. Kechel 11:51, 3. Nov 2005 (CET) 
\item Projekt teamfound auf berlios.de beantragt. Kechel 00:27, 3. Nov 2005 (CET) 
\end{itemize}

\paragraph{02. November 2005}

\begin{itemize}
\item TeamFound Wiki-Seite grunds�tzlich strukturiert, Teilprojekte angelegt und �berall mal einen Anfang formuliert. Kechel 23:00, 2. Nov 2005 (CET) 
\end{itemize}

\paragraph{01. November 2005}

\begin{itemize}
\item Nachtrag: Gemeinsames Treffen an Uni, Festlegung der Projekte, und verteilung der Projektmitglieder (so kamen wir 4 erstmals zu diesem Projekt zusammen). Kechel 23:00, 2. Nov 2005 (CET) 
\end{itemize}


\section{Firefox Toolbar Changelog}
\label{changelog}
Das originale \index{Changelog}Changelog\footnote{\texttt{http://teamfound.berlios.de/\#firefoxchangelog0.7}} (in englischer Sprache):
\subsection{2006}
Bisher noch kein neues Release in 2006. Die version 0.8 ist zwar bereits eine vollst"andige Imlementation nach Mileston 2, funktioniert aber noch nicht stabil genug um diese als neues Release freizugeben.
\subsection{2005}
\subsubsection{Firefox \& Flock toolbar 0.7 (07-DEC-2005)}
\begin{itemize}
\item Bug-Fix: Add-Button actually works now
\item Label with current URL removed (though we are not going to replace the location-bar)
\item Compatibility with Flock browser tested and added to install.rdf
\end{itemize}

\subsubsection{Firefox toolbar 0.6 (30-NOV-2005)}

\begin{itemize}
\item Input-field now behaves the same like the normal Firefox locationbar (including history popup) when urls are typed. Soon this can replace the Firefox locationbar and the Firefox searchbar with only one input-field ;-)
\item Settings-Dialog allows to give custom external search engines
\item Settings-Dialog allows to only search external, only search TeamFound or search both at once
\end{itemize}

\subsubsection{Firefox toolbar 0.5 (28-NOV-2005)}

\begin{itemize}
\item Google search results are actually working now, including further results
\item Compatible with Firefox 1.5 release candidates
\end{itemize}

\subsubsection{Firefox toolbar 0.4 (27-NOV-2005)}

\begin{itemize}
\item Clicking on TF-Icon opens preferences dialog, here you can setup your TeamFound server url and choose between two search result layouts
\item Add page - Adds the current page to the TeamFound-Index.
\item text field - Here you can enter search words or urls. The toolbar tries to automatically figure out what you meant and though will either search or just visit the given url.
\item current url - The current url is always displayed at the end of the toolbar, though not overwriting your current search words.
\end{itemize}

\subsubsection{Firefox toolbar 0.3 (24-NOV-2005)}

\begin{itemize}
\item adds search at extern search-engine
\item shows google and teamfound search-results in same window
\end{itemize}

\subsubsection{Firefox toolbar 0.2 (22-NOV-2005)}

\begin{itemize}
\item first release at berlios.de
\item available as .xpi
\end{itemize}

\subsubsection{Firefox toolbar 0.1 (19-NOV-2005)}

\begin{itemize}
\item initial release
\end{itemize}

\chapter{TeamFound Programm Lizenz}
TeamFound sowie alle enthaltenen Komponenten wie Source-Code, Binaries und Bilder stehen under der GNU General Public License (GPL) Version 2 oder h"oher. Eine Kopie der Lizenz liegt allen Source- und Bin"ar Versionen bei.\\
\\
\textbf{TeamFound - share your search results}\\
\textbf{Copyright \copyright 2005-2006 Jan Kechel, Martin Klink, Jonas Heese, Andreas Bachmann}\\
\\
This program is free software; you can redistribute it and/or
modify it under the terms of the GNU General Public License
as published by the Free Software Foundation; either version 2
of the License, or (at your option) any later version.
This program is distributed in the hope that it will be useful,
but WITHOUT ANY WARRANTY; without even the implied warranty of
MERCHANTABILITY or FITNESS FOR A PARTICULAR PURPOSE. See the
GNU General Public License for more details.
You should have received a copy of the GNU General Public License
along with this program; if not, write to the Free Software
Foundation, Inc., 51 Franklin Street, Fifth Floor, Boston, MA 02110-1301, USA.

\chapter{GNU Free Documentation License}
%\label{label_fdl}

 \begin{center}

       Version 1.2, November 2002


 Copyright \copyright 2000,2001,2002  Free Software Foundation, Inc.
 
 \bigskip
 
     51 Franklin St, Fifth Floor, Boston, MA  02110-1301  USA
  
 \bigskip
 
 Everyone is permitted to copy and distribute verbatim copies
 of this license document, but changing it is not allowed.
\end{center}


\begin{center}
{\bf\large Preamble}
\end{center}

The purpose of this License is to make a manual, textbook, or other
functional and useful document ''free'' in the sense of freedom: to
assure everyone the effective freedom to copy and redistribute it,
with or without modifying it, either commercially or noncommercially.
Secondarily, this License preserves for the author and publisher a way
to get credit for their work, while not being considered responsible
for modifications made by others.

This License is a kind of ''copyleft'', which means that derivative
works of the document must themselves be free in the same sense.  It
complements the GNU General Public License, which is a copyleft
license designed for free software.

We have designed this License in order to use it for manuals for free
software, because free software needs free documentation: a free
program should come with manuals providing the same freedoms that the
software does.  But this License is not limited to software manuals;
it can be used for any textual work, regardless of subject matter or
whether it is published as a printed book.  We recommend this License
principally for works whose purpose is instruction or reference.


\begin{center}
{\Large\bf 1. APPLICABILITY AND DEFINITIONS}
\addcontentsline{toc}{section}{1. APPLICABILITY AND DEFINITIONS}
\end{center}

This License applies to any manual or other work, in any medium, that
contains a notice placed by the copyright holder saying it can be
distributed under the terms of this License.  Such a notice grants a
world-wide, royalty-free license, unlimited in duration, to use that
work under the conditions stated herein.  The \textbf{''Document''}, below,
refers to any such manual or work.  Any member of the public is a
licensee, and is addressed as \textbf{''you''}.  You accept the license if you
copy, modify or distribute the work in a way requiring permission
under copyright law.

A \textbf{''Modified Version''} of the Document means any work containing the
Document or a portion of it, either copied verbatim, or with
modifications and/or translated into another language.

A \textbf{''Secondary Section''} is a named appendix or a front-matter section of
the Document that deals exclusively with the relationship of the
publishers or authors of the Document to the Document's overall subject
(or to related matters) and contains nothing that could fall directly
within that overall subject.  (Thus, if the Document is in part a
textbook of mathematics, a Secondary Section may not explain any
mathematics.)  The relationship could be a matter of historical
connection with the subject or with related matters, or of legal,
commercial, philosophical, ethical or political position regarding
them.

The \textbf{''Invariant Sections''} are certain Secondary Sections whose titles
are designated, as being those of Invariant Sections, in the notice
that says that the Document is released under this License.  If a
section does not fit the above definition of Secondary then it is not
allowed to be designated as Invariant.  The Document may contain zero
Invariant Sections.  If the Document does not identify any Invariant
Sections then there are none.

The \textbf{''Cover Texts''} are certain short passages of text that are listed,
as Front-Cover Texts or Back-Cover Texts, in the notice that says that
the Document is released under this License.  A Front-Cover Text may
be at most 5 words, and a Back-Cover Text may be at most 25 words.

A \textbf{''Transparent''} copy of the Document means a machine-readable copy,
represented in a format whose specification is available to the
general public, that is suitable for revising the document
straightforwardly with generic text editors or (for images composed of
pixels) generic paint programs or (for drawings) some widely available
drawing editor, and that is suitable for input to text formatters or
for automatic translation to a variety of formats suitable for input
to text formatters.  A copy made in an otherwise Transparent file
format whose markup, or absence of markup, has been arranged to thwart
or discourage subsequent modification by readers is not Transparent.
An image format is not Transparent if used for any substantial amount
of text.  A copy that is not ''Transparent'' is called \textbf{''Opaque''}.

Examples of suitable formats for Transparent copies include plain
ASCII without markup, Texinfo input format, LaTeX input format, SGML
or XML using a publicly available DTD, and standard-conforming simple
HTML, PostScript or PDF designed for human modification.  Examples of
transparent image formats include PNG, XCF and JPG.  Opaque formats
include proprietary formats that can be read and edited only by
proprietary word processors, SGML or XML for which the DTD and/or
processing tools are not generally available, and the
machine-generated HTML, PostScript or PDF produced by some word
processors for output purposes only.

The \textbf{''Title Page''} means, for a printed book, the title page itself,
plus such following pages as are needed to hold, legibly, the material
this License requires to appear in the title page.  For works in
formats which do not have any title page as such, ''Title Page'' means
the text near the most prominent appearance of the work's title,
preceding the beginning of the body of the text.

A section \textbf{''Entitled XYZ''} means a named subunit of the Document whose
title either is precisely XYZ or contains XYZ in parentheses following
text that translates XYZ in another language.  (Here XYZ stands for a
specific section name mentioned below, such as \textbf{''Acknowledgements''},
\textbf{''Dedications''}, \textbf{''Endorsements''}, or \textbf{''History''}.)  
To \textbf{''Preserve the Title''}
of such a section when you modify the Document means that it remains a
section ''Entitled XYZ'' according to this definition.

The Document may include Warranty Disclaimers next to the notice which
states that this License applies to the Document.  These Warranty
Disclaimers are considered to be included by reference in this
License, but only as regards disclaiming warranties: any other
implication that these Warranty Disclaimers may have is void and has
no effect on the meaning of this License.


\begin{center}
{\Large\bf 2. VERBATIM COPYING}
\addcontentsline{toc}{section}{2. VERBATIM COPYING}
\end{center}

You may copy and distribute the Document in any medium, either
commercially or noncommercially, provided that this License, the
copyright notices, and the license notice saying this License applies
to the Document are reproduced in all copies, and that you add no other
conditions whatsoever to those of this License.  You may not use
technical measures to obstruct or control the reading or further
copying of the copies you make or distribute.  However, you may accept
compensation in exchange for copies.  If you distribute a large enough
number of copies you must also follow the conditions in section 3.

You may also lend copies, under the same conditions stated above, and
you may publicly display copies.


\begin{center}
{\Large\bf 3. COPYING IN QUANTITY}
\addcontentsline{toc}{section}{3. COPYING IN QUANTITY}
\end{center}


If you publish printed copies (or copies in media that commonly have
printed covers) of the Document, numbering more than 100, and the
Document's license notice requires Cover Texts, you must enclose the
copies in covers that carry, clearly and legibly, all these Cover
Texts: Front-Cover Texts on the front cover, and Back-Cover Texts on
the back cover.  Both covers must also clearly and legibly identify
you as the publisher of these copies.  The front cover must present
the full title with all words of the title equally prominent and
visible.  You may add other material on the covers in addition.
Copying with changes limited to the covers, as long as they preserve
the title of the Document and satisfy these conditions, can be treated
as verbatim copying in other respects.

If the required texts for either cover are too voluminous to fit
legibly, you should put the first ones listed (as many as fit
reasonably) on the actual cover, and continue the rest onto adjacent
pages.

If you publish or distribute Opaque copies of the Document numbering
more than 100, you must either include a machine-readable Transparent
copy along with each Opaque copy, or state in or with each Opaque copy
a computer-network location from which the general network-using
public has access to download using public-standard network protocols
a complete Transparent copy of the Document, free of added material.
If you use the latter option, you must take reasonably prudent steps,
when you begin distribution of Opaque copies in quantity, to ensure
that this Transparent copy will remain thus accessible at the stated
location until at least one year after the last time you distribute an
Opaque copy (directly or through your agents or retailers) of that
edition to the public.

It is requested, but not required, that you contact the authors of the
Document well before redistributing any large number of copies, to give
them a chance to provide you with an updated version of the Document.


\begin{center}
{\Large\bf 4. MODIFICATIONS}
\addcontentsline{toc}{section}{4. MODIFICATIONS}
\end{center}

You may copy and distribute a Modified Version of the Document under
the conditions of sections 2 and 3 above, provided that you release
the Modified Version under precisely this License, with the Modified
Version filling the role of the Document, thus licensing distribution
and modification of the Modified Version to whoever possesses a copy
of it.  In addition, you must do these things in the Modified Version:

\begin{itemize}
\item[A.] 
   Use in the Title Page (and on the covers, if any) a title distinct
   from that of the Document, and from those of previous versions
   (which should, if there were any, be listed in the History section
   of the Document).  You may use the same title as a previous version
   if the original publisher of that version gives permission.
   
\item[B.]
   List on the Title Page, as authors, one or more persons or entities
   responsible for authorship of the modifications in the Modified
   Version, together with at least five of the principal authors of the
   Document (all of its principal authors, if it has fewer than five),
   unless they release you from this requirement.
   
\item[C.]
   State on the Title page the name of the publisher of the
   Modified Version, as the publisher.
   
\item[D.]
   Preserve all the copyright notices of the Document.
   
\item[E.]
   Add an appropriate copyright notice for your modifications
   adjacent to the other copyright notices.
   
\item[F.]
   Include, immediately after the copyright notices, a license notice
   giving the public permission to use the Modified Version under the
   terms of this License, in the form shown in the Addendum below.
   
\item[G.]
   Preserve in that license notice the full lists of Invariant Sections
   and required Cover Texts given in the Document's license notice.
   
\item[H.]
   Include an unaltered copy of this License.
   
\item[I.]
   Preserve the section Entitled ''History'', Preserve its Title, and add
   to it an item stating at least the title, year, new authors, and
   publisher of the Modified Version as given on the Title Page.  If
   there is no section Entitled ''History'' in the Document, create one
   stating the title, year, authors, and publisher of the Document as
   given on its Title Page, then add an item describing the Modified
   Version as stated in the previous sentence.
   
\item[J.]
   Preserve the network location, if any, given in the Document for
   public access to a Transparent copy of the Document, and likewise
   the network locations given in the Document for previous versions
   it was based on.  These may be placed in the ''History'' section.
   You may omit a network location for a work that was published at
   least four years before the Document itself, or if the original
   publisher of the version it refers to gives permission.
   
\item[K.]
   For any section Entitled ''Acknowledgements'' or ''Dedications'',
   Preserve the Title of the section, and preserve in the section all
   the substance and tone of each of the contributor acknowledgements
   and/or dedications given therein.
   
\item[L.]
   Preserve all the Invariant Sections of the Document,
   unaltered in their text and in their titles.  Section numbers
   or the equivalent are not considered part of the section titles.
   
\item[M.]
   Delete any section Entitled ''Endorsements''.  Such a section
   may not be included in the Modified Version.
   
\item[N.]
   Do not retitle any existing section to be Entitled ''Endorsements''
   or to conflict in title with any Invariant Section.
   
\item[O.]
   Preserve any Warranty Disclaimers.
\end{itemize}

If the Modified Version includes new front-matter sections or
appendices that qualify as Secondary Sections and contain no material
copied from the Document, you may at your option designate some or all
of these sections as invariant.  To do this, add their titles to the
list of Invariant Sections in the Modified Version's license notice.
These titles must be distinct from any other section titles.

You may add a section Entitled ''Endorsements'', provided it contains
nothing but endorsements of your Modified Version by various
parties--for example, statements of peer review or that the text has
been approved by an organization as the authoritative definition of a
standard.

You may add a passage of up to five words as a Front-Cover Text, and a
passage of up to 25 words as a Back-Cover Text, to the end of the list
of Cover Texts in the Modified Version.  Only one passage of
Front-Cover Text and one of Back-Cover Text may be added by (or
through arrangements made by) any one entity.  If the Document already
includes a cover text for the same cover, previously added by you or
by arrangement made by the same entity you are acting on behalf of,
you may not add another; but you may replace the old one, on explicit
permission from the previous publisher that added the old one.

The author(s) and publisher(s) of the Document do not by this License
give permission to use their names for publicity for or to assert or
imply endorsement of any Modified Version.


\begin{center}
{\Large\bf 5. COMBINING DOCUMENTS}
\addcontentsline{toc}{section}{5. COMBINING DOCUMENTS}
\end{center}


You may combine the Document with other documents released under this
License, under the terms defined in section 4 above for modified
versions, provided that you include in the combination all of the
Invariant Sections of all of the original documents, unmodified, and
list them all as Invariant Sections of your combined work in its
license notice, and that you preserve all their Warranty Disclaimers.

The combined work need only contain one copy of this License, and
multiple identical Invariant Sections may be replaced with a single
copy.  If there are multiple Invariant Sections with the same name but
different contents, make the title of each such section unique by
adding at the end of it, in parentheses, the name of the original
author or publisher of that section if known, or else a unique number.
Make the same adjustment to the section titles in the list of
Invariant Sections in the license notice of the combined work.

In the combination, you must combine any sections Entitled ''History''
in the various original documents, forming one section Entitled
''History''; likewise combine any sections Entitled ''Acknowledgements'',
and any sections Entitled ''Dedications''.  You must delete all sections
Entitled ''Endorsements''.

\begin{center}
{\Large\bf 6. COLLECTIONS OF DOCUMENTS}
\addcontentsline{toc}{section}{6. COLLECTIONS OF DOCUMENTS}
\end{center}

You may make a collection consisting of the Document and other documents
released under this License, and replace the individual copies of this
License in the various documents with a single copy that is included in
the collection, provided that you follow the rules of this License for
verbatim copying of each of the documents in all other respects.

You may extract a single document from such a collection, and distribute
it individually under this License, provided you insert a copy of this
License into the extracted document, and follow this License in all
other respects regarding verbatim copying of that document.


\begin{center}
{\Large\bf 7. AGGREGATION WITH INDEPENDENT WORKS}
\addcontentsline{toc}{section}{7. AGGREGATION WITH INDEPENDENT WORKS}
\end{center}


A compilation of the Document or its derivatives with other separate
and independent documents or works, in or on a volume of a storage or
distribution medium, is called an ''aggregate'' if the copyright
resulting from the compilation is not used to limit the legal rights
of the compilation's users beyond what the individual works permit.
When the Document is included in an aggregate, this License does not
apply to the other works in the aggregate which are not themselves
derivative works of the Document.

If the Cover Text requirement of section 3 is applicable to these
copies of the Document, then if the Document is less than one half of
the entire aggregate, the Document's Cover Texts may be placed on
covers that bracket the Document within the aggregate, or the
electronic equivalent of covers if the Document is in electronic form.
Otherwise they must appear on printed covers that bracket the whole
aggregate.


\begin{center}
{\Large\bf 8. TRANSLATION}
\addcontentsline{toc}{section}{8. TRANSLATION}
\end{center}


Translation is considered a kind of modification, so you may
distribute translations of the Document under the terms of section 4.
Replacing Invariant Sections with translations requires special
permission from their copyright holders, but you may include
translations of some or all Invariant Sections in addition to the
original versions of these Invariant Sections.  You may include a
translation of this License, and all the license notices in the
Document, and any Warranty Disclaimers, provided that you also include
the original English version of this License and the original versions
of those notices and disclaimers.  In case of a disagreement between
the translation and the original version of this License or a notice
or disclaimer, the original version will prevail.

If a section in the Document is Entitled ''Acknowledgements'',
''Dedications'', or ''History'', the requirement (section 4) to Preserve
its Title (section 1) will typically require changing the actual
title.


\begin{center}
{\Large\bf 9. TERMINATION}
\addcontentsline{toc}{section}{9. TERMINATION}
\end{center}


You may not copy, modify, sublicense, or distribute the Document except
as expressly provided for under this License.  Any other attempt to
copy, modify, sublicense or distribute the Document is void, and will
automatically terminate your rights under this License.  However,
parties who have received copies, or rights, from you under this
License will not have their licenses terminated so long as such
parties remain in full compliance.


\begin{center}
{\Large\bf 10. FUTURE REVISIONS OF THIS LICENSE}
\addcontentsline{toc}{section}{10. FUTURE REVISIONS OF THIS LICENSE}
\end{center}


The Free Software Foundation may publish new, revised versions
of the GNU Free Documentation License from time to time.  Such new
versions will be similar in spirit to the present version, but may
differ in detail to address new problems or concerns.  See
http://www.gnu.org/copyleft/.

Each version of the License is given a distinguishing version number.
If the Document specifies that a particular numbered version of this
License ''or any later version'' applies to it, you have the option of
following the terms and conditions either of that specified version or
of any later version that has been published (not as a draft) by the
Free Software Foundation.  If the Document does not specify a version
number of this License, you may choose any version ever published (not
as a draft) by the Free Software Foundation.


\begin{center}
{\Large\bf ADDENDUM: How to use this License for your documents}
\addcontentsline{toc}{section}{ADDENDUM: How to use this License for your documents}
\end{center}

To use this License in a document you have written, include a copy of
the License in the document and put the following copyright and
license notices just after the title page:

\bigskip
\begin{quote}
    Copyright \copyright  YEAR  YOUR NAME.
    Permission is granted to copy, distribute and/or modify this document
    under the terms of the GNU Free Documentation License, Version 1.2
    or any later version published by the Free Software Foundation;
    with no Invariant Sections, no Front-Cover Texts, and no Back-Cover Texts.
    A copy of the license is included in the section entitled ''GNU
    Free Documentation License''.
\end{quote}
\bigskip
    
If you have Invariant Sections, Front-Cover Texts and Back-Cover Texts,
replace the ''with...Texts.'' line with this:

\bigskip
\begin{quote}
    with the Invariant Sections being LIST THEIR TITLES, with the
    Front-Cover Texts being LIST, and with the Back-Cover Texts being LIST.
\end{quote}
\bigskip
    
If you have Invariant Sections without Cover Texts, or some other
combination of the three, merge those two alternatives to suit the
situation.

If your document contains nontrivial examples of program code, we
recommend releasing these examples in parallel under your choice of
free software license, such as the GNU General Public License,
to permit their use in free software.



\begin{thebibliography}{keine ah}
\addcontentsline{toc}{chapter}{Literaturverzeichnis}

\bibitem[Apa02]{} \emph{Apache Ant 1.5.1 Manual}. Apache Software Foundation, http://ant.apache.org/manual/, 2002.
\bibitem[Dat00]{} C. J. Date. \emph{An Introduction to Database Systems}. Addison Wesley, 2000.
\bibitem[ExFF]{} \emph{Extend Firefox}. Mozilla, http://developer.mozilla.org/mozilla-org/contests/extendfirefox/documentation.php
\bibitem[FFTT]{} \emph{Firefox Toolbar Tutorial}. Born Geek, http://www.borngeek.com/firefox/toolbar-tutorial/
\bibitem[Fri97]{} Jeffrey E. F. Friedl. \emph{Mastering Regular Expressions}. O'Reilly, 1997.
\bibitem[Gswed]{} \emph{Getting started with extension development}. mozillaZine, http://kb.mozillazine.org/Getting\_started\_with\_extension\_development.
\bibitem[Goo00]{} M. Goossens, F. Mittelbach und A. Samarin. \emph{Der \LaTeX \ Begleiter}. Addison Wesley, 2000.
\bibitem[LucQS]{} \emph{Apache Lucene Query Syntax}. Apache Software Foundation http://lucene.apache.org/java/docs/queryparsersyntax.html .
\bibitem[Oua02]{} S. Oualline. \emph{Vi IMproved -- Vim}. New Riders, 2002 .
\bibitem[PolPos]{} \emph{PolePosition}. the open source database benchmark http://www.polepos.org.
\bibitem[Roac04]{} Eric. \emph{How to create Firefox extensions}. roachfiend.com, http://roachfiend.com/archives/2004/12/08/how-to-create-firefox-extensions
\bibitem[Spo01]{} J. Spolsky. \emph{User Interface Desing for Programmers}. Apress, 2001 .
\bibitem[XULPL]{} \emph{XUL Planet} XUL Planet, http://xulplanet.com .

\end{thebibliography}

