
\chapter{Projektverlauf}
\section{Organisation}
Gleich zu Anfang des Projekts haben wir uns auf feste Kommunikationswege geeinigt. Diese bildeten, neben regelm"a"sigen Treffen an der Uni, die Basis unserers Entwicklungsprozesses.
\subsection{Kommunikation}
\begin{enumerate}
\item Entwicklungs-Arbeit zu Standards, Interfaces, Realisierungen etc. finden auf Wiki-Seite \texttt{http://wiki.jonasheese.de/index.php/TeamFound} statt.

\item Mailingliste \texttt{teamfound-development@lists.berlios.de} - zum subscriben/unsubscriben folgenden Link benutzen:\\
\texttt{https://lists.berlios.de/mailman/listinfo/teamfound-development}

\item Diskussionen bez"uglich der Vorlesung finden auf der ossi Mailingliste \texttt{ossi@insel.cs.tu-berlin.de} mit Subject: TeamFound: xxx statt.
\end{enumerate}

\subsection{Source-Code Versions Verwaltung}
Um mit mehreren Personen gleichzeitig an gleichen Teilen Source-Codes zu Arbeiten, wird eine Sourcecode-Verwaltung ben"otigt. Es gibt viele verschiedene Internet-Seiten, die ,f"ur freie Software, kostenlos einen solchen Service anbieten. Wir haben uns f"ur Berlios (\texttt{http://developer.berlios.de/}) entschieden, da hier im Gegensatz zu Sourceforge (\texttt{http://sourceforge.net}) SVN (Subversion) anstelle von CVS (Concurrent Version Control) angeboten wird. Berlios beschreibt sich selber auf seiner Homepage als:

\textit{BerliOS Developer ist ein freier Dienst f�r Open Source Entwickler und bietet einfachen Zugang zum Besten aus CVS/SVN, Mailinglisten, Bug-Tracking, Diskussionsforen, Aufgabenverwaltung, Webhosting, dauerhafte Dateiarchivierung, Backups und vollst�ndige Verwaltung per Web-Interface.}\\
\\
Die Details des TeamFound-Projekts auf Berlios sind:
\begin{description}
\item[Project page] https://developer.berlios.de/projects/teamfound
\item[Project full name] TeamFound
\item[Project unix name] teamfound
\item[SVN server] svn.berlios.de\footnote{Details zum SVN: \texttt{https://developer.berlios.de/svn/?group\_id=5199}}
\item[Shell server] shell.berlios.de
\item[Web server] http://teamfound.berlios.de
\end{description}

\subsection{Teilprojekte}
Als zweite, grundlegende organisatorische Entwicklungsplattform, haben wir uns f"ur ein MediaWiki entschieden, welches Jonas freundlicherweise auf einem seiner Server aufgesetzt hatte. Um die Entwicklung in klare Bahnen zu lenken, haben wir dort verschiedene Teilprojekte erstellt, die mehr oder wenig unabh"angig voneinander bearbeitet werden konnten. Eher weniger von einander unabh"angig war und ist besonders das Teilprojekt Interface-Spezifikation. Dieses wurde bei beginn eines neuen Milestones jeweils als erstes Ausgearbeitet, damit die Server- und Client-Implementationen zeitgleich durchgef"uhrt werden konnten.
\begin{description}
\item[Milestones] Aktueller Stand: Milestone 1 fertig, Milestone 2 fast fertig

\item[Toolbar als Extension f�r Firefox] Aktueller Stand: Toolbar 0.7 funktionsfaehig (Milestone 1), Toolbar Milestone 2 fast fertig

\item[Toolbar als Extension f�r Internet Explorer] Aktueller Stand: Toolbar 0.1 fertig (Milestone 1)

\item[Server] Aktueller Stand: version 0.1 einsatzbereit (Milestone 1)

\item[Interface-Spezifikation] Aktueller Stand: Interfaces fuer Milestone 1 fertig, Milestone 2 fertig 

\item[Pr�sentation] Aktueller Stand: Enduser-Homepage angefangen, Logo vorgeschlagen 

\item[Link als Client] Aktueller Stand: Milestone 1 fertig: http://hqpm.dyndns.org/tf/ 

\item[Kategorien] Aktueller Stand: noch auszuarbeiten 

\end{description}
\section{Meilensteine}
