\chapter{Neue Features}
\section{Neuerungen am Server}

\subsection{Web-Crawler}

Eine wichtige Komponente in einer Suchmaschine ist der Crawler, welcher die zu indexierenden Dokumente herunterl�dt. Die bisherige recht einfache Implementierung hat diesen Job nur teilweise gut erledigt. So war es bisher nicht m�glich die Codierung der heruntergeladenen Daten korrekt zu bestimmen. Au�erdem konnte Teamfound nicht mit Framesets umgehen, bzw. hat versucht die Datei mit dem Frameset zu indexieren, nicht den Inhalt der eigentlichen Frames.
 
Um diese Probleme zu l�sen wurde der Teamfound-Crawler komplett neu implementiert. Eine dreistufige Suche nach einer Codierungsangabe, n�mlich in der Verbindung an sich, im HTTP-Header sowie im HTML-Header. Letzterer Ort ist der warscheinlichste Fundort f�r ein solches Encoding. Es ist daf�r also notwendig den kompletten HTML-Code herunterzuladen und dann nach einer entsprechenden Angabe zu suchen. Die Suche nach Angaben zum Encoding sieht die M�glichkeit vor, das Teamfoudn in der Zukunft mehrere Arten von Dokumenten indexieren kann. Da die Suche nach einem Encoding in PDF-Dateien zB. ganz anders ablaufen wird, muss diese vom Inhaltstyp abh�ngen. Der neue Crawler l�sst sich bequem um solche erweitern.

Die Behandlung von Framesets ist eine zentrale Anforderung, die auf verschiedene Arten gel�st werden kann. So ist die Frage, ob einfach alle Frames in der Reihenfolge ihrer Definition in der Quelldatei heruntergeladen und zu einem langen Dokument zusammengef�gt werden, oder ob mehr Aufwand betrieben werden soll, um zB. den Hauptinhaltsframe zu finden und an den Anfang des Ergebnisdokumentes zu stellen. Die neue Implementierung des Crawler verfolgt erstes Schema, alle Frames werden in einfacher Definitionsreihenfolge hintereinander kopiert und dann an den Indexer �bergeben, welcher das HTML entfernt und den Text indexiert. So steht aber nat�rlich nicht immer der Anfang des Inhaltes in der Teamfound-Ergebnisanzeige. Da es aber nicht einfach ist, den Hauptinhaltsframe innerhalb eines Framesets herauszufinden haben wir uns entschieden dieses auch nicht zu versuchen.

\subsection{Index-Updates}

Wurden bisher Internetseiten in einen Teamfound-Server aufgenommen so wurden sie w�hrend dieses Vorganges vom Crawler heruntergeladen und in den Index gespeichert. �nderungen an der Seite w�rden sp�ter nicht in den Index aufgenommen werden da bereits eingetragene Seiten nicht erneuert wurden.
Um dieses Problem aus der Welt zu schaffen startet ein Teamfound-Server nun einen nebenl�ufigen Thread der alle Seiten in einem bestimmten Intervall erneut herunterl�dt und bei �nderungen neu indexiert.

Dabei k�nnen Seiten, wenn der Crawler auf HTTP-Fehler st��t, vorr�bergehend deaktiviert werden, oder wenn diese Fehler anhalten, auch komplett aus dem Index entfernt werden.

\subsection{Logging}

Da der Funktionsumfang des Teamfound-Servers stark angewachsen ist, wurde die Nutzung eines echten Logging-Systems notwendig, einerseits vereinfacht dies die Entwicklung, da Debug-Ausgaben ordentlich ausgegeben werden k�nnen, andererseits ist es f�r einen sp�teren Produktiven Einsatz evtl. wichtig Fehler bei Zugriffen auf den Teamfound-Server oder fehlgeschlagene Logins zu protokollieren.
Da w�hrend eines produktiven Einsatzes keine oder nur wenige Debug-Nachrichten protokolliert werden sollen, ist ein Logging-System notwendig, das einerseits zwischen der Art von Nachrichten unterscheiden kann, andererseits aber auch Komponenten, welche Nachrichten einliefern, trennen kann. log4j aus dem Apache-Projekt erf�llt alle diese Aufgaben und wurde daher f�r die Integration in Teamfound ausgew�hlt. Einzelne Komponenten benutzen unterschiedliche sogenannte 'Logger' und trennen ihre Nachrichten in mehrere Kategorien zwischen Debug, Info sowie Warning und Error auf. Einzelne oder alle Logger k�nnen auf die Konsole oder in eine Datei umgeleitet werden.  

\section{Neuerungen am Web-Interface}
\textit{Neuerungen} ist gut, bisher gab es eigentlich noch kein Web-Interface :)

Die Version 0.2 des TeamFound Servers hatte noch die Option \texttt{want=xml|html}, "uber die man vom Server direkt eine im Browser darstellbare HTML-Seite anfordern konnte. Jedoch war dieser Teil des Servers zu grossen Teilen nicht implementiert.

In der aktuellen Version 0.3 setzen wir nur auf XML-Antworten, die daf"ur aber allesamt ein XSL-Transformations Stylesheet enthalten:

\texttt{<?xml-stylesheet href=$``$transform.xsl$``$ type=$``$text/xsl$``$?>}

"Uber diese XSLT Datei k"onnen alle g"angigen Browser das vom Server gelieferte XML selber in HTML umwandeln und anzeigen.

\subsection{Komponenten}
Das gesamte Web-Interface besteht aus 3 Dateien:
\begin{description}
\item[transform.xsl] Transformiert unseren XML-Baum in HTML.
\item[stylesheet.css] Legt das Layout des erstellten HTMLs fest.
\item[ui.js] Enth"alt eine einzige JavaScript Funktion (\texttt{showhide(layer\_ref)}) zum Anzeigen/Verstecken einzelner Bereiche der Webseite.
\end{description}

Die \texttt{transform.xsl} ist die einzige dieser Dateien die ich hier genauer beschreiben will. 



\section{Neuerungen an der Firefox/Flock-Toolbar}
\section{Neuerungen an der Internet-Explorer-Toolbar}

