
\chapter{Projektverlauf}
\section{Organisation und Kommunikation by Jonas}

Das Teamfound-Projekt hat gleich mit zwei sehr starken Belastungen zu k�mpfen, einerseits ist es ein OpenSource-Projekt und kann zwar auf ordentliche Werkzeuge wie Revisionsverwaltung und Mailinglisten zur�ckgreifen, lebt jedoch von lockerer Organisation und der weitgehend freiwilligen Mitarbeit aller Beteiligten, andererseits ist es prim�r immernoch ein Projekt an der Universit�t und hat damit mit den ganz normalen Problemen solcher Projekte zu k�mpfen. Dies schliesst inbesondere den ganz eigenen Arbeitsablauf in derartigen Projekten ein, also mehr Aktivit�t, je n�her der Abgabetermin f�r diesen Bericht r�ckt.

Anders als im letzten Semester gab es dieses mal keine regelm��igen Treffen. Abgesehen von einem Starttreffen, bei dem vorallem Ideen f�r Erweiterungen und Bugs ausgetauscht wurden und einem Treffen kurz vor Schluss, in welchem die noch zahlreichen offenen Baustellen besprochen wurden, gab es kein Treffen von allen beteiligten. Benutzt wurde daher, wie auch im letzten Semester schon, die Mailingliste des Projektes (teamfound-development �t lists.berlios.de), um sich auszutauschen und Ideen und vorgehensweisen zu besprechen. Weitere Werkzeuge, wie sie Berlios anbietet, darunter Feature- und Bugtracker, wurden wohl auch aufgrund des geringen nutzens f�r ein derart kleines Projekt noch nicht benutzt. Gegen Ende des Projektes sind auch die ungeplanten treffen im inoffiziellen Projekt-IRC-Channel (\#teamfound - quakenet) h�ufiger und bekannt f�r deren Produktivit�t geworden.

\subsection{Fazit: Probleme wachsender Projekte}

Teamfound besteht mittlerweile aus einer Menge Code, so sind alleine im Server mittlerweile an die einhundert Klassen mit vielen tausend Zeilen Code implementiert. In diesem Semester sind ausserdem viele Funktionen hinzugekommen, so das sich die Aktionsimplementierungen im Server fast verdoppelt haben. Es zeigt sich nun das die bisherige Architektur kaum mehr geeigent ist um mit den Anforderungen klarzukommen, so sind die zentralen Punkte im Server, n�mlich das Servlet, welches Requests entgegennimmt, validiert undn an den Controller weiterleitet, sowie dieser Controller, der die eigentliche Nutzlast f�r die einzelnen Aktionen implementiert,v�llig �berlaufen.
In Zukunft wird es also immer wichtiger effektive Strukturen f�r diese wachsenden Anforderungen zu schaffen und zu dokumentieren. Klare Datenschnittstellen sind notwendig und bisher kaum umgesetzt, die Kapselung von Funktionalit�t ist eine Aufgabe, welche die Architektur des Teamfound-Servers noch einmal deutlich ver�ndern wird. Die Auslagerung von Code in einzelne Bibliotheken-gleiche Implementierungen, wie zB. mit dem Crawler geschehen, k�nnen eine M�glichkeit sein, besser wart- und testbaren Code zu schreiben und auch f�r eine sauberere Implementierung sorgen. 

\section{Projektverlauf by Jan}

W"ahrend des Semesters hat sich eigentlich garnichts zu diesem Projekt getan. Selbst Steffen habe ich irgendwann im Juli 2006 gefragt ob wir uns nicht mal wieder Treffen sollten. Am 23. August 2006 war es dann soweit, und wir haben gemeinsam "uber m"ogliche Erweiterungen und neue Features diskutiert. Prinzipiell kam dabei aber nichts rum, was nicht schon lange (seit 15. Mai 2006) im wiki unter Milestone 3 eingetragen war. Die dort genannten Features f"ur den Server entsprechen lustigerweise auch genau den jetzt implementierten: 
\begin{verbatim}
http://wiki.jonasheese.de/index.php/Milestones

Server 17:05, 15. Mai 2006 Kechel (->Milestone 3 (SS06) - Planung)

* XSL-Stylesheet anstelle von HTML-Antwort von Server 
  -> es wird immer in XML geantwortet
* Web-Admin-Interface
* Framebasierte Seiten unterst�tzen
* User-Management
* Zeichensaetze beachten
* Automatisches Updated der eingetragenen Seiten (alle x tage oder so) 
\end{verbatim}

Inwieweit das mit den Zeichens"atzen tats"achlich funktioniert weiss ich pers"onlich nicht, ich habe aber im Source-Code gesehen das Jonas irgendwas mit ''encoding'' und so gemacht hat.

Nach diesem Treffen passierte wiederum erstmal eine Woche lang garnichts.

Ich begann dann am 2. September in unserem wiki das Interface f"ur Milestone 3 (\texttt{http://wiki.jonasheese.de/index.php/Interface\_Milestone\_3}) zu spezifizieren. Dazu "uberlegten Martin und ich welche Funktionen denn f"ur die obigen Features ben"otigt w"urden, welche Parameter sinnvoll seien und wie die XML-Antwort aussehen k"onnte. Leider hat Jonas's Server immernoch als Datum den 1. Januar 1970, daher ist die History dieses Dokuments nicht so eindeutig zur"uckzuverfolgen. F"ur sehr verl"asslich halte ich aber die Angaben in unserem Logbuch (\texttt{http://wiki.jonasheese.de/index.php/Logbuch}). Jedenfalls ist diese \textit{Spezifikation}, ausgedruckt auf DIN A4 Papier, 12 Seiten lang, und wurde im weiteren Projektverlauf von allen Projektseiten wohl am h"aufigsten benutzt und erweitert. Die einzig andere, f"ur mich n"utzliche Seite im Wiki, war \texttt{Server Milestone 3}, da Martin hier die Datenbank Tabellen incl. der Spaltennamen und Fremdschl"ussel aufgef"uhrt hat.

Am 6. September war wieder ein gemeinsames Treffen, jedoch waren nur Steffen, Andreas und Ich anwesend. Nach etwa zwei Stunden palaber war auch das geschafft. Wir hatten zwar interessante Themen, jedoch nicht wirklich spezifisch "uber das TeamFound Projekt diskutiert. 

Nach diesen Vorbereitungen war ich eine Woche im Ausland, mit dem Bewusstsein da"s nur nach meiner R"uckkehr nur noch eine einzige Woche Zeit verbleiben w"urde um "uberhaupt etwas zu implementieren. Dementsprechend vermisste ich auch sehr die nicht vorhandene Aktivit"at auf unserer Mailingliste. Nicht eine einzige Mail bzgl. TeamFound erreichte mich in dieser Zeit.  

Zum Gl"uck hatte sich doch etwas getan. Jedenfalls hatte Martin viel am DBLayer gearbeitet, und somit die Infrastruktur f"ur unsere neuen Features stark vorangebracht. Auch Jonas arbeitete schon am Index und Crawler. Was genau da alles ver"andert wurde oder werden sollte weiss ich nicht, aber bis jetzt hatte ich auch noch keine einzige Zeile am Sever selber geschrieben und wusste dementsprechend wenig dar"uber.

Hand in Hand mit Martin begann ich dann am Server zu arbeiten, und ein Feature nach dem anderen zu Implementieren. Da wir Zimmernachbarn sind, war die Kommunikation sehr sehr einfach und produktiv. Ich musste einfach nur \textit{r"ubergehen} und sagen $``$\textit{Hey, schalt mal deinen Chat-Client an, ich hab da ne Frage.}$``$ "Uber die folgenden Tage lernte ich so sehr schnell mich im Server-Code zurechtzufinden. Nebenbei erstellte ich noch das Web-Interface und passte die Firefox-Toolbar an das neue Interface an (ca. eine Stunde FF-Toolbar). Das Web-Interface entwickelte sich genauso wie Martin und ich Funktionen im Server fertigstellten, und nutzten dieses auch immer sofort um die neuen Funktionen zu testen. Jonas hatte derweil den neuen Indexer und Crawler fertiggestellt. Nach ein paar Stunden Bugfixing von Martin war diese Komponente auch schon voll funktionsf"ahig (Jonas hatte anscheinend Probleme mit seinem Internetzugang, und konnte daher bis jetzt seine "Anderungen nicht comitten. Es ergaben sich auch viele Merge-Konflikte durch diese Verz"ogerung..).

Gestern, zwei Tage vor Abgabetermin, hatten wir endlich alle Features, bis auf das L"oschen von Kategorien und das automatische Updaten der indizierten Seiten, fertiggestellt, und ich konnte einen "offentlichen Testserver aufsetzen (\texttt{http://teamfound.dyndns.org:8080/tf/tf}). Im Chat kl"arten wir mit Andreas auch die letzten Interface-Fragen f"ur seine IE Toolbar implementation. Bis dahin hatte ich null Ahnung ob Andreas "uberhaupt an seinem Teil des Projekts arbeitet. Wie sich herausstellte ist seine Toolbar jedoch inzwischen wesentlich weiter als \textit{meine}. Immerhin half ich ihm einen Fehler beim Einloggen zu l"osen, eine Funktion die in der FF-Toolbar noch v"ollig fehlt.

Danach erstellte ich ein neues Release v0.9 der FF-Toolbar auf berlios.de und passte unsere Projekt Web-Seite \texttt{http://teamfound.berlios.de} entsprechend an.

Heute, einen Tag vor Abgabe, h"orte ich soeben von Martin, da"s die Update-Funktionalit"at nun auch funktioniert (w"ahrend ich selber diesen Text hier schreibe). Es sieht also so aus als w"urde alles \textit{just in time} fertig werden :)\\


Als Fazit w"urde ich sagen, da"s besonders die Interface Spezifikation sehr sehr Hilfreich war, um die zu implementierenden Features genau festzulegen. Auch die v"ollig von mir getrennt abgelaufene Entwicklung der Internet Explorer Toolbar wurde sicherlich nur dadurch m"oglich.

Innerhalb des Servers fehlten mir eben diese gemeinsamen Strukturen. Wie ich in der letzten Woche gelernt habe gibt es zwar auch dort viele feste Vereinbarungen, jedoch habe ich erst jetzt, nach einer Woche intensiver Besch"aftigung damit, den groben "Uberblick dar"uber erhalten. Ich bin sicher da"s Martin und Jonas diesen "Uberblick schon immer hatten, aber ich vermisste eine kurze Beschreibung z.B. im wiki dazu doch sehr. 

Es f"allt auch auf das keinerlei Coding-Styles vereinbart wurden. Es war mir nicht m"oglich Funktionsnamen aufgrund eines festgelegten Schemas zu erraten oder \textit{schnell} zu finden. Meist hab ich einfach Martin gefragt, und er hat mir dann gesagt ob es die Funktion, die xyz tut, schon gibt, oder ob ich diese neu Implementieren muss.

Zum Schluss noch die Anmerkung, da"s unser manuelles Logfile im wiki praktisch nur von mir alleine geupdated wurde. Ich hab zwar Martin immer wieder mal dazu "uberreden k"onnen, auch einzutragen was er denn so gemacht hat, aber das Interesse daran war wohl bei niemandem sehr hoch. Mir pers"onlich hat das Log besonders f"ur diesen Projektbericht sehr geholfen, aber f"ur die eigentliche Projektentwicklung war es nicht notwendig. Dennoch finde ich es selber sehr sch"on den Projektverlauf so zur"uckverfolgen zu k"onnen.\\

So, jetzt bin ich selber gespannt ob wir heute noch den Server v0.3 released bekommen.\\

Jan

\section{Projektverlauf von Martin}
Der Start des Projekts war einerseits gut, da wir alle 4 weiter an dem Projekt arbeiten wollten.
Was bedeutet, dass es wohl vorher f"ur keinen von uns schlecht gelaufen ist.

Leider begannen wir auch mit der halboffiziellen "Ubereinkunft, dass wir w"ahrend der Vorlesungszeit alle
keine Zeit auf das Projekt verwenden k"onnen. Andere Universit"are Leistungen und Pr"ufungen bzw.
Arbeit oder "ahnliches waren dringender. Ich war selbst damit sehr Einverstanden, da ich
auch nicht gewusst h"atte wann ich noch Zeit er"ubrigen kann.

Das bringt nat"urlich wenn es dann losgehen soll wieder Einarbeitungszeit mit sich. Hinzu kommt f"ur
mich, dass vieles was ich schon am ende des letzten Semester als TODO auf der Liste hatte immer
noch unver"andert ist. Eher ist die Liste l"anger geworden.

Trozdem sehe ich den Verlauf des Projekts positiv.
Bei unserem ersten gemeinsamen Treffen haben wir festgelegt welche Erweiterungen wir noch in
Teamfound einbauen wollen. Jan hat ziemlich schnell ein Grundger"ust 
f"urs Interface im Wiki angelegt. Die Interfacespezifikation war in diesem wie auch 
im vorigen Semester eines der wichtigsten Hilfsmittel. 
Ich konnte immer gut vergleichen was schon funktioniert
und was noch fehlt. 
Ausserdem sind n"otige "Anderungen und Erweiterungen, die einer erarbeitet sofort f"ur alle
nachlesbar.
Einen grossteil der n"otigen Request/Response Kommunikation
hatten wir beim ersten Treffen besprochen, allerdings
war die Spezifikation am Anfang noch sehr rudiment"ar. Wir hatten erstmal nur die 
Aufrufparameter festgelegt und Jan hatte ein leicht ver"andertes XML-Grundger"ust f"ur die
Serverantwort entworfen.

Deshalb habe ich zu beginn immer gleich f"ur jede Funktion, die ich implementiert
habe auch die Spezifikation um die Serverantwort erweitert. 
Das ist sp"ater leider aufgrund von Zeitdruck unter den Tisch gefallen.

Bis zum letzten Treffen vor Abgabe verlief die Arbeit ziemlich z"ah. Danach daf"ur wirklich
gut. Alle waren wieder oft im IRC-Channel. Die schnellen Absprachen auf diese Art boten
neben dem Wiki aus meiner Sicht die beste Kommunikationsm"oglichkeit  w"ahrend der Implementation.
Ich habe gut davon profitiert, weil die andern mir in vielen Belangen doch ein
St"uck voraus sind und helfen konnten.
Die Mailingliste lief da eher schlecht, weil meist einfach keine direkte Antwort da ist und
man schlechter diskutieren kann.

Um ein Fazit zu ziehen, w"urde ich sagen es ist ausser Startschwierigkeiten gut gelaufen 
und wir haben alles umgesetzt was wir uns vorgenommen hatten. 
Leider sind wir noch kein richtiges OpenSource Projekt, da wir den Abgabedruck brauchen um weiterzukommen anstatt von alleine den Antrieb zu entwickeln unser Projekt voranzutreiben. 





