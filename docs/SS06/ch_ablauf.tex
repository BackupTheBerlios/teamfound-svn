
\chapter{Projektverlauf}
\section{Organisation und Kommunikation}

Das Teamfound-Projekt hat gleich mit zwei sehr starken Belastungen zu k�mpfen, einerseits ist es ein Open Source Projekt und kann zwar auf ordentliche Werkzeuge wie Revisionsverwaltung und Mailinglisten zur�ckgreifen, lebt jedoch von lockerer Organisation und der weitgehend freiwilligen Mitarbeit aller Beteiligten, andererseits ist es prim�r immernoch ein Projekt an der Universit�t und hat damit mit den ganz normalen Problemen solcher Projekte zu k�mpfen. Dies schliesst inbesondere den ganz eigenen Arbeitsablauf in derartigen Projekten ein, also mehr Aktivit�t, je n�her der Abgabetermin f�r diesen Bericht r�ckt.

Anders als im letzten Semester gab es dieses mal keine regelm��igen Treffen. Abgesehen von einem Starttreffen, bei dem vorallem Ideen f�r Erweiterungen und Bugs ausgetauscht wurden und einem Treffen kurz vor Schluss, in welchem die noch zahlreichen offenen Baustellen besprochen wurden, gab es kein Treffen von allen beteiligten. Benutzt wurde daher, wie auch im letzten Semester schon, die Mailingliste des Projektes (teamfound-development �t lists.berlios.de), um sich auszutauschen und Ideen und vorgehensweisen zu besprechen. Weitere Werkzeuge, wie sie Berlios anbietet, darunter Feature- und Bugtracker, wurden wohl auch aufgrudn des geringen nutzens f�r ein derart kleines Projekt noch nicht benutzt. Gegen Ende des Projektes sind auch die ungeplanten treffen im inoffiziellen Projekt-IRC-Channel (#teamfound - quakenet) h�ufiger und bekannt f�r deren Produktivit�t geworden.